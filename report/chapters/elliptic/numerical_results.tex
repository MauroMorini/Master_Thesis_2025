% Numerical Results ----------------------------------------------------------------------------------------------------------------------------------------------------------------
\section{Numerical Results}
\subsection{Rate of Convergence}
\label{subsec:conv_rate_ell}
First we would like to test the code we have written in \texttt{MATLAB} and try to replicate the convergence rates provided in section \ref{sec:elliptic_error_analysis}. 
The procedure of doing so can be listed in the following steps: \\

\begin{stepscope}
    \begin{step}[Domain]
    We fix a domain $\Omega$. In our case $\Omega = (0,1)$ stays unchanged for all the experiments of this chapter.
    \end{step}

    \begin{step}[Exact Solution]
        We decide a priori which exact solution we intend to approximate. To simplify calculations we choose one of the simple smooth solutions:
        \begin{equation*}
            u_1(x) = x^r, \qquad u_2(x) = e^{-x}\sin(5x),
        \end{equation*}
        where $r$ corresponds to the polynomial degree of the finite element space. \\
        $u_1$ is chosen to ensure the exactness of the method (since we approximate a polynomial of degree $r$ by a piecewise polynomial of degree $r$
        the finite element approximation should be exact up to floating-point errors).
    \end{step}

    \begin{step}[Coefficient]
        We fix a smooth coefficient, one of 
        \begin{equation*}
            c_1(x) = 1, \qquad c_2(x) = \sin(x) + 2,
        \end{equation*}
        where having these two options allows us to first check if the method works for a homogeneous background material and then for the inhomogeneous case.
        To ensure our exact solution still satisfies the original pde, we calculate the forcing term $f$ in (\ref{eq:elliptic_pde}) symbolically, i.e.\ 
        $f_{i,j} = -(c_j \, u_i^{\prime})^{\prime} $.
    \end{step}

    \begin{step}[Boundary Conditions]
        We choose the type of boundary condition we would like to impose on each boundary. We have mainly focused on Dirichlet boundary conditions on both sides so far, so this is 
        what we will consider. Let it be said, that the results for one Neumann and one Dirichlet (as mentioned in section \ref{sec:boundary_conditions_elliptic}) are very similar. \\
        To actually impose the boundary condition we directly take the values of the previously fixed exact solution $u$ at the respective boundary points, meaning 
        \begin{equation*}
            g_{0,i} = u_i(0), \qquad g_{1,i} = u_i(1).
        \end{equation*}
    \end{step}

    \begin{step}[Mesh]
        We choose an initial mesh, i.e.\ an initial partition of the domain $\Omega$. For our purpose we only present the equidistant mesh
        \begin{equation*}
           \mathcal{T}_h^{(0)} = \{(0, 0.25), (0.25, 0.5),(0.5, 0.75),(0.75, 1)\}, 
        \end{equation*}
        with initial meshsize $h = \frac{1}{4}$, but the results do not vary for any kind of initial mesh. In addition to the initial mesh we decide on a number of refinement cycles. Refining each
        element of the initial mesh by dividing it into two new elements and repeating this process for the number of refinement cycles fixed yields a sequence of nested meshes. 
    \end{step}

    \begin{step}[Numerical Approximation]
        Here we only present the case for $r \in \{1,2\}$, meaning for $\mathcal{P}^1$-, $\mathcal{P}^2$-elements, 
        but the expected convergence rates can be recovered for an arbitrary polynomial degree
        $r$. For $r>6$ there is barely a convergence rate to be made out anymore since after only a couple of refinements the error flattens out and even starts to grow slightly. 
        This happens due to the high condition number of the resulting stiffness matrix $\mathbf{B}$ and the high number of global degrees of freedom (basis nodes). \\
        We also fix a big enough penalization parameter $\sigma = 10(r+1)^2$ to ensure coercivity. \\ \\
        On each successive mesh $\mathcal{T}_h^{(l)}$ we find the numerical solution $ u_h^{(l)}$ by solving the system (\ref{eq:fully_discrete_dg_system_elliptic}).
    \end{step}

    \begin{step}[Error Computation]
        We are interested in 
        \begin{enumerate}
            \item the $L^2$-error
                \begin{equation*}
                    \|u - u_h\|_{L^2(\Omega)} = \Big( \int_{\Omega} |u(x) - u_h(x)|^2 \text{d} x \Big)^{1/2} = \Big( \sum_{n=0}^{N}\int_{I_n} |u(x) - u_h(x)|^2 \text{d} x \Big)^{1/2};
                \end{equation*}
            \item the broken $H^1$-error
                \begin{equation*}
                    \|u - u_h\|_{H^1(\mathcal{T}_h )} = \Big( \sum_{n=0}^{N}\int_{I_n} |u^{\prime}(x) - u_h^{\prime}(x)|^2 \text{d} x \Big)^{1/2};
                \end{equation*}
            \item and finally the error in the energy norm 
                \begin{equation*}
                    \|u - u_h\|_{\epsilon} = \Big( \sum_{n=0}^{N} \int_{I_n} c(x)\abs{u^{\prime}(x) - u_h^{\prime}(x)}^2\text{d}x + 
                    \sum_{n=0}^{N+1}{\normalfont \texttt{a}_n}\jump{u(x_n) - u_h(x_n)}^2 \Big)^{1/2}.
                \end{equation*}
        \end{enumerate}
        We compute for each numerical solution $ u_h^{(l)} $ all three named errors and plot them in a plot where both axes are logarithmically scaled.
    \end{step}
\end{stepscope}

First to ensure exactness we consider the polynomial exact solution $u_1$ with the constant coefficient $c_1$, as expected 
all three errors display minimal floating-point values of around $10^{-15}$, 
slightly increasing with each further refinement due to the growing condition number of the system matrix. 
When considering $u_1$ with the non-trivial coefficient $c_2$ we do now get real decreasing error rates due to the introduced quadrature error when
assembling the stiffness matrix and the load vector. 

\begin{figure}[h!]
    \centering
    
    \begin{minipage}[t]{0.44\textwidth}
        \centering
        \includegraphics[width=\linewidth]{figures/dg_elliptic_uniform_mesh_exact_sol_errors_P1.jpg}
        \caption{Errors of SIPG for $P^1$-elements}
        \label{fig:elliptic_uniform_mesh_error_p1}
    \end{minipage}
    \hfill
    \begin{minipage}[t]{0.44\textwidth}
        \centering
        \includegraphics[width=\linewidth]{figures/dg_elliptic_uniform_mesh_exact_sol_errors_P2.jpg}
        \caption{Errors of SIPG for $P^2$-elements}
        \label{fig:elliptic_uniform_mesh_error_p2}
    \end{minipage}
\end{figure}

In the Figures \ref{fig:elliptic_uniform_mesh_error_p1}, \ref{fig:elliptic_uniform_mesh_error_p2} we observe the expected convergence rates of 
$ \mathcal{O}(h^r)$ in the broken $H^1$- and energy norm as well as $ \mathcal{O}(h^{r+1})$ in the $L^2$-norm for the exact solution
$u_2(x) = e^{-x}\sin(5x)$ and the coefficient $c_2(x) = \sin(x) + 2$. We note that the error in the energy norm and the error in the broken $H^1$-norm behave equivalently.
Furthermore we observe that the error in the case of $P^2$-elements flattens out and starts to grow again 
after crossing the threshold of $10^{-10} $, which as mentioned before is to be expected. In fact an error of $10^{-10}$ is way smaller than one 
hopes to achieve in reality using standard computational tools. \\ 
If instead we choose the constant coefficient $c_1(x) = 1$, the picture is the same. \\ \\

\begin{figure}[h!]
    \centering
    
    \begin{minipage}[t]{0.44\textwidth}
        \centering
        \includegraphics[width=\linewidth]{figures/dg_elliptic_num_sol_p1.jpg}
        \caption{high penalty SIPG-approximation for $P^1$-elements}
        \label{fig:elliptic_uniform_mesh_sol_p1}
    \end{minipage}
    \hfill
    \begin{minipage}[t]{0.44\textwidth}
        \centering
        \includegraphics[width=\linewidth]{figures/dg_elliptic_num_sol_p2.jpg}
        \caption{high penalty SIPG-approximation for $P^2$-elements}
        \label{fig:elliptic_uniform_mesh_sol_p2}
    \end{minipage}
\end{figure}

\subsection{Visualization of the SIPG Solution}
Now we visualize the numerical solution approximating $u_2$ on the uniform mesh $\mathcal{T}_h^{(1)}$ (after one global refinement cycle) and play with the penalization
parameter $ \sigma $. If we penalize the jumps heavily by using the generously chosen $\sigma = 10(r+1)^2$ we can not observe any discontinuity in the numerical solution
by eye (see figures \ref{fig:elliptic_uniform_mesh_sol_p1} and \ref{fig:elliptic_uniform_mesh_sol_p2}). Also the boundary conditions seem to be exact, as they would be in the case of 
continuous FEM, of course this is not the case as we will see shortly.
The circles in the figures \ref{fig:elliptic_uniform_mesh_sol_p1} and \ref{fig:elliptic_uniform_mesh_sol_p2} denote the face nodes of the elements in $\mathcal{T}_h^{(1)}$ (meaning the element boudnary points), 
in between each of those face nodes the numerical solution is a polynomial of degree $r$, which when comparing $r=1$ and $r=2$ as done here, becomes apparent. \\

\begin{figure}[h!]
    \centering
    
    \begin{minipage}[t]{0.44\textwidth}
        \centering
        \includegraphics[width=\linewidth]{figures/dg_elliptic_num_sol_non_coercive_p1.jpg}
        \caption{non-coercive SIPG-approximation for $P^1$-elements}
        \label{fig:elliptic_uniform_mesh_sol_p1_non_coercive}
    \end{minipage}
    \hfill
    \begin{minipage}[t]{0.44\textwidth}
        \centering
        \includegraphics[width=\linewidth]{figures/dg_elliptic_num_sol_non_coercive_p2.jpg}
        \caption{non-coercive SIPG-approximation for $P^2$-elements}
        \label{fig:elliptic_uniform_mesh_sol_p2_non_coercive}
    \end{minipage}
\end{figure}


If we relax the penalty by reducing the penalization parameter enough such that the bilinear form 
is no longer coercive, we can observe the jumps growing. If we look at the figures
\ref{fig:elliptic_uniform_mesh_sol_p1_non_coercive}, \ref{fig:elliptic_uniform_mesh_sol_p2_non_coercive}, where we have set $\sigma = 1.1$, the discontinuities become apparent.
The bilinear form loosing it's coercivity means that the system we solve in (\ref{eq:fully_discrete_dg_system_elliptic}) is not invertible anymore and 
we end up solving a least squares problem. Here we have chosen extreme edge examples of $\sigma$ on purpose to illustrate the effect of modulating the penalization parameter and to visually show
the discontinuity of a DG-solution.
Maybe interesting to observe is the effect of enforcing boundary conditions weakly. Clearly here the Dirichlet boundary condition $u(0) = 0$ is 
not fulfilled by the numerical solutions, which would be the case for strongly enforced boundary conditions. \\


\subsection{Influence of the Quadrature Rule on the Convergence Rate}
\label{sec:elliptic_numerical_experiments}
As noted in section \ref{sec:quadrature_rule} by using $r+1$ Gauss-Lobatto nodes (in the context of $\mathcal{P}^r$-elements) as quadrature nodes and basis nodes at the same time
introduces an error when assembling the mass matrix of the system. In this subsection we experimentally compare the results of using $r+1$ Gauss-Lobatto quadrature nodes
to approximate the integrals versus using the exact integration values. To be precise we fix a polynomial degree $r$ and our Lagrangian basis with $r+1$ Gauss-Lobatto nodes
as specified in section \ref{sec:ell_basis} and approximate all integrals first using the Gauss-Lobatto quadrature rule with $r+1$ nodes, then with $r+2$ nodes and compare the two.
\\
First we have to consider a slightly different elliptical problem, which requires a mass matrix. 
\begin{equation}
	\label{eq:elliptic_pde_with:mass}
	-(c(x)u'(x))' + u(x) = f(x) \qquad \forall x\in \Omega \nonumber,
\end{equation}
\begin{equation}
	\label{eq:elliptic_pde_bc_with_mass}
	u(0) = g_0, u(1) = g_1 \nonumber,
\end{equation}
We can apply the exact same tools as in the derivation of the variational formulation in section \ref{sec:elliptic_var_form} and get the discrete SIPG variational formulation. \\
Find $u_h \in V_h$ such that:
\begin{equation}
	\label{eq:discrete_var_form_elliptic_with_mass}
	b_h(u_h, v) + (u_h, v)_{L^2(\Omega)} = \ell_h(v), \qquad \forall v\in V_h,
\end{equation}
Now analogously to section \ref{sec:matrix_vect_syst} we write $u_h = \sum_{m=0}^{N} \sum_{j=0}^r \alpha_j^m \Phi_j^m \in V_h$ and find that (\ref{eq:discrete_var_form_elliptic_with_mass})
is equivalent to \\
\begin{equation*}
	\sum_{m=0}^{N} \sum_{j=0}^r \alpha_j^m \Big(b_h(\Phi_j^m, \Phi_i^n) + (\Phi_j^m, \Phi_i^n)_{L^2(\Omega)}\Big) = \ell_h(\Phi_i^n) \qquad \forall i \in \{0,\ldots,r\}, n \in \{0,\ldots,N\},
\end{equation*}
which in turn is equivalent to the Matrix-Vector system

\begin{equation}
	\label{eq:elliptic_matrix_vect_system_with_mass}
	\Big( \textbf{B} + \textbf{M} \Big) \textbf{u} = \textbf{l},
\end{equation}
where as before 
$ [\textbf{B}]_{T(n,i), T(m,j)} = b_h(\Phi_j^m, \Phi_i^n),
	[\textbf{u}]_{T(m,j)} = \alpha_j^m,
	[\textbf{l}]_{T(n,i)} = \ell_h(\Phi_i^n)$ and furthermore
$ [\textbf{M}]_{T(n,i), T(m,j)} = (\Phi_j^m, \Phi_i^n)_{L^2(\Omega)}$.
\\ \\
We now consider a similar setting as described in subsection \ref{subsec:conv_rate_ell} and test the convergence rates for $\mathcal{P}^1,\mathcal{P}^2$-elements,
where we choose the exact solution $u_2(x) = e^{-x}\sin(5x)$ and the strongly oscillating coefficient $c_3(x) = \sin(20x) + 2$. 
We distinguish the following two methods of calculating numerical solutions.

\begin{figure}[h!]
    \centering
    \begin{minipage}[t]{0.44\textwidth}
        \centering
        \includegraphics[width=\linewidth]{figures/dg_error_elliptic_with_and_without_quad_p1.jpg}
        \caption{Comparison of convergence rates of higher order vs lower order quadrature for $\mathcal{P}^1$-elements}
        \label{fig:elliptic_with_and_without_quad_conv_rates_p1}
    \end{minipage}
    \hfill
    \begin{minipage}[t]{0.44\textwidth}
        \centering
        \includegraphics[width=\linewidth]{figures/dg_error_elliptic_with_and_without_quad_p2.jpg}
        \caption{Comparison of convergence rates of higher order vs lower order quadrature for $\mathcal{P}^2$-elements}
        \label{fig:elliptic_with_and_without_quad_conv_rates_p2}
    \end{minipage}
\end{figure}

\subsubsection*{Method A: "Low Order"}
We assemble the matrices of the system (\ref{eq:elliptic_matrix_vect_system_with_mass})
as described in section \ref{sec:stiff_assembly} and approximate the integrals appearing in the entries of \textbf{A}, \textbf{M}, \textbf{l} 
using the (lower order) $r+1$ node Gauss-Lobatto quadrature rule, where $r \in \{1,2\}$, meaning the quadrature nodes coincide with the basis nodes
and the mass matrix is mass-lumped, hence the name.
We calculate the $L^2$-, $H^1(\mathcal{T}_h)$- and energy error between the numerical solution and the exact solution on all meshes.

\subsubsection*{Method B: "High Order"}
We assemble the matrices of the system (\ref{eq:elliptic_matrix_vect_system_with_mass})
as described in section \ref{sec:stiff_assembly} and approximate the integrals appearing in the entries of \textbf{A}, \textbf{M}, \textbf{l} 
using the (higher order) $r+2$ node Gauss-Lobatto quadrature rule, where $r \in \{1,2\}$, meaning the mass matrix is calculated exactly.
We calculate the $L^2$-, $H^1(\mathcal{T}_h)$- and energy error between the numerical solution and the exact solution on all meshes.
\\ \\
It turns out that there is only a very slight difference in accuracy, plotting the convergence rates of both methods into the same plot 
does not yield a very informative picture as is visible in Figures 
\ref{fig:elliptic_with_and_without_quad_conv_rates_p1}, \ref{fig:elliptic_with_and_without_quad_conv_rates_p2}. 
Instead since we expect the error resulting from Method B to be smaller than the error from Method A, we subtract
Error B from Error A and plot this difference into a loglog plot, see Figures 
\ref{fig:elliptic_with_and_without_quad_error_difference_p1}, \ref{fig:elliptic_with_and_without_quad_error_difference_p2}.

\begin{figure}[h!]
    \centering
    \begin{minipage}[t]{0.44\textwidth}
        \centering
        \includegraphics[width=\linewidth]{figures/dg_error_elliptic_with_and_without_quad_error_difference_p1.jpg}
        \caption{Lower order error minus higher order error for $\mathcal{P}^1$-elements}
        \label{fig:elliptic_with_and_without_quad_error_difference_p1}
    \end{minipage}
    \hfill
    \begin{minipage}[t]{0.44\textwidth}
        \centering
        \includegraphics[width=\linewidth]{figures/dg_error_elliptic_with_and_without_quad_error_difference_p2.jpg}
        \caption{{Lower order error minus higher order error for $\mathcal{P}^2$-elements}}
        \label{fig:elliptic_with_and_without_quad_error_difference_p2}
    \end{minipage}
\end{figure}

We observe that the convergence rate is not affected by using a mass lumping strategy. Clearly we only gain a slight bit of accuracy by employing a 
higher order quadrature rule. This speaks for using mass lumping when needed. On the other hand since we already have a block-diagonal mass matrix 
in DG there is really no benefit to using mass lumping, we simply sacrifice accuracy. In a more realistic application we might be more interested
in accuracy than convergence rates and especially if the forcing term $f$ or the coefficient $c$ rapidly change in a certain part of the domain,  
loosing out on accuracy might be unnecessary. In practice it is the norm to use a higher order quadrature rule if possible.

\subsection{Modeling an Inhomogeneous Membrane}
To round off this chapter we now present a small intuitive example of the pde (\ref{eq:elliptic_pde})-(\ref{eq:elliptic_pde_bc}). This subsection is inspired by the step-5 tutorial program
of the \texttt{deal.II} \footnote{www.dealii.org} website. \\
Imagine a thin one dimensional membrane of a certain homogeneous material being spanned over the domain $\Omega = (0,1)$, such that the left end is fixed at the point $x = 0$ to a height of $y = 0$ 
and the right end is fixed at the point $x = 1$ to a height of $y = 0$. Since the membrane is in a relaxed state and pinned to a fixed height at both ends the picture we get is just a flat membrane.
We can describe the vertical displacement of the membrane by a function $u : \Omega \to \R$. This function is harmonic with zero Dirichlet boundary conditions in the relaxed state, so 
the displacement satisfies $u \equiv 0$. \\ \\
Now we can introduce a constant force $f \equiv 1$ on  $\Omega$ pushing the membrane up. The displacement of the membrane is now a solution of the Poisson equation 
\begin{equation*}
    -u^{\prime \prime} = f \qquad  \text{in } \Omega,
\end{equation*} 
\begin{equation*}
    u(0) = u(1) = 0.
\end{equation*}
As a final modification we allow for an inhomogeneous stiffness of the membrane, for example by letting its thickness vary over $\Omega$. The stiffness is described by 
the coefficient 
\begin{equation*}
    c(x) = 
        \begin{cases}
            1, & x \in \left[0, 0.3 \right) \cup \left(0.7, 1\right] \\
            20, & x \in [0.3, 0.7]
        \end{cases}
\end{equation*} 
so that the membrane is twenty times thicker in the center region than closer to the boundary. Now the displacement $u$ is a solution of the problem 
(\ref{eq:elliptic_pde})-(\ref{eq:elliptic_pde_bc}) with $g_0 = g_1 = 0$ and $f \equiv 1$. \\ \\
We discretize the domain into a non equidistant partition, where we allow for smaller elements in the region of higher stiffness around $[0.3, 0.7]$.
We simulate the displacement $u$ with $\mathcal{P}^2$-elements using the SIPG variational form. 

\begin{figure}[h!]
    \centering
    \begin{minipage}[t]{0.3\textwidth}
        \vspace{0pt}
        In Figure \ref{fig:elliptic_simulation_inhomogeneous_membrane} we observe the characteristic shape of a pressurized membrane corresponging to a solution of the Poisson equation. The 
        inhomogeneity in the center, where the membrane displays higher stiffness, barely deforms in response to the pressure introduced by $f$, this yields a flattened plateau in 
        the interval $[0.3, 0.7]$.
    \end{minipage}
    \hfill
    \begin{minipage}[t]{0.6\textwidth}
        \vspace{0pt}
        \centering
        \includegraphics[width=\linewidth]{figures/ellitpic_simulation_inhomogeneous_membrane.jpg}
        \caption{Displacement of inhomogeneous membrane with $\mathcal{P}^2$-elements}
        \label{fig:elliptic_simulation_inhomogeneous_membrane}
    \end{minipage}
\end{figure}
