\section{Absorbing Boundary Conditions}
\label{sec:transparent_bc}
When simulating a wave a classical example of a domain on which to do so is a \textit{wave guide}, a rectangular domain through which the 
wave is propagated. Since we restrict ourselves to the 1d case in this thesis, trivially every domain interval we can choose corresponds to 
a sort of waveguide. We are interested in simulating a wave which propagates through this waveguide only coming from one direction. 
Lets say we want to send a wave through the waveguide originating from the lower boundary of the domain. We can achieve that by imposing 
a corresponding Dirichlet or Neumann boundary condition at the point of entry. A question arises: \\ \\
\textit{What boundary condition should be imposed at the upper boundary?} \\ \\
The answer may differ depending on the purpose of the simulation. The problem with using a simple Dirichlet or Neumann boundary condition at the upper boundary
as well is that the incident wave is directly influenced and modified by the upper boundary condition. 
Lets say we impose a Dirichlet boundary condition at the lower boundary and a homogeneous Neumann boundary condition at the upper boundary
\begin{equation*}
    u(0) = \sin(x-t),  \qquad u_x(1) = 0,
\end{equation*}  
then the upper boundary will reflect the full wave back. This clearly pollutes the incident wave and if we wanted to observe how the the incident wave changes 
by passing through an inhomogeneous medium, the reflected wave would overshadow any slight modulation stemming from variations in the material.
In reality in this context we are interested in simulating the problem on an unbounded domain. We would like the wave to just propagate through the upper boundary
and not return. 
To simulate an artificial boundary of this kind we need to impose artificial boundary conditions, these are often called \textit{absorbing-, transparent-,} or 
\textit{non-reflecting} boundary conditions. \\ \\
We impose the first order absorbing boundary condition 
\begin{equation}
    \label{eq:absorbing_bc}
    \frac{\partial u}{\partial t} +  \sqrt{c} \frac{\partial u}{\partial n} = 0
\end{equation}
at the upper boundary (exit of the waveguide), which corresponds to 
\begin{equation*}
    u_t(10, \cdot) + \sqrt{c(10,\cdot)} u_x(10,\cdot) = 0 \qquad \text{in } [0, T].
\end{equation*}

To implement this boundary condition into our time-marching scheme we have to first treat the bilinear form $b$ at the upper boundary as if we were to implement 
a Neumann boundary condition (see section \ref{sec:boundary_conditions_elliptic}), meaning after integrating by parts and summing over all elements we don't introduce
the symmetry and penalty terms at the corresponding node $x_{N+1}$, instead we apply the absorbing boundary condition and get
\begin{equation*}
    -\{c(x_{N+1},t) \partial_x u_h(x_{N+1},t)\}[v(x_{N+1})] = \{\sqrt{c(x_{N+1},t)} \partial_t u_h(x_{N+1},t) \} [v(x_{N+1})].
\end{equation*}

the semi-discrete scheme (\ref{eq:wave_semidiscrete_var_form}) turns into
\begin{equation*}
    (\partial_{t}^2 u_h(t), v)_{L^2(\Omega)} + b(u_h,v;t) + \{\sqrt{c(x_{N+1},t)} \partial_t u_h(x_{N+1},t) \} [v(x_{N+1})] = \ell (v;t) 
    \qquad \forall v \in V_h, t \in [0,T],
\end{equation*}
where 
\begin{align*}
    b(u,v;t) & = \sum_{n=0}^N \int_{I_n} c(x,t)u_x(x,t)v_x(x)\, \text{d}x
		-\sum_{n=0}^{N} \{c(x_n,t)u_x(x_n, t)\}[v(x_n)] + \{c(x_n,t)v_x(x_n)\}[u(x_n,t)] \\
		& + \sum_{n=0}^{N} \texttt{a}_n(t)[u(x_n,t)][v(x_n)],                                   \\
		\ell(v;t)  & = (f(t),v)_{L^2(\Omega)}+ g_0(t)c(x_0^+,t)v_x(x_0^+) 
	    + \texttt{a}_0(t) g_0(t) v(x_{0}^+)
\end{align*}
are slightly modified. \\
And the semi-discrete matrix-vector system becomes
\begin{equation*}
     \textbf{M} \ddot{\mathbf{u}}(t) + \mathbf{B}(t)\mathbf{u}(t) + \mathbf{R}(t)\dot{\mathbf{u}}(t) = \mathbf{l}(t) \qquad \forall t \in  [0,T],
\end{equation*}
where $[\mathbf{R}(t)]_{i,j} = \{\sqrt{c(x_{N+1},t)} \Phi_j(x_{N+1},t) \} [\Phi_i(x_{N+1})]$ is a matrix where the only non-zero entry is the one at the bottom right corner
where $i = j = \text{dim}(V_h)$ and $\Phi_i$ the basis centered at $x_{N+1}$ is.
\\ \\
From here on we can use a (second order) centered finite difference to approximate  
\begin{equation*}
    \dot{\mathbf{u}}(t) \approx \frac{\mathbf{u}(t + \Delta t) - 
    \mathbf{u}(t - \Delta t)}{2 \Delta t} 
\end{equation*}
and similarly to before we find the slightly modified, fully-discrete leapfrog scheme
\begin{equation*}
    \Big(\mathbf{M} + \frac{\Delta t}{2}\mathbf{R}_m\Big) \mathbf{u}_{m+1} = \Delta t^2 \mathbf{l}_m + (2 \mathbf{M} - \Delta t^2 \mathbf{B}_m) \mathbf{u}_{m} 
    + \Big(\frac{\Delta t}{2}\mathbf{R}_m - \mathbf{I}\Big) \mathbf{u}_{m-1}.
\end{equation*}
We have to also make a slight adjustment to the initialization of the solution at the first time-step $\mathbf{u}_1$. By using a Taylor-expansion and 
the semi-discrete scheme we have just derived we can define
\begin{equation*}
    \mathbf{u}_1 :=   \mathbf{u}_0 + \Delta t \mathbf{v}_0 + 
    \frac{\Delta t^2}{2}\mathbf{M}^{-1}( \mathbf{l}_0 - \mathbf{B}_0 \mathbf{u}_0 - \mathbf{R}_0 \mathbf{v}_0).
\end{equation*}  