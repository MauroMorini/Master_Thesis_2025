\chapter{DG for the Wave Equation}
In the last chapter we have derived the necessary theoretical tools and built code to solve the time-independent
elliptic problem. This elliptic problem coincides with a time-independent version of the hyperbolic problem
\begin{equation*}
    u_{tt} - (c u_x)_x = f.
\end{equation*} 
We will use a \textit{method of lines} approach to first discretize in space with discontinuous Galerkin finite elements, 
using what we have derived in the last chapter, and
then discretize in time using leapfrog time-integration to solve the problem numerically. This chapter is structured similarly to the last one.
First we pose the problem and recall some analysis on regularity and existence, uniqueness of solutions. 
Next we discretize both in time and space deriving the fully discrete scheme. We will then again elaborate on the actual implementation of the program 
as well as recalling convergence theory. Finally we will present the numerical experiments reproducing the expected convergence rates where 
exact solutions are given and simulate the behavior of waves propagating through a waveguide with material properties varying in space and time. 

% Problem ----------------------------------------------------------------------------------------------------------------------------------------------------
\section{Problem}
Let $\Omega = (0,1)$ be the domain (waveguide) and let $T > 0$ be the endtime. The 1d wave equation with a time-dependent coefficient is given by 
\begin{subequations} \label{eq:strong_form_wave}
    \begin{align}
        u_{tt}(x,t) - (c(x,t)u_x(x,t))_x &= f(x,t)  &&\forall (x,t) \in \Omega \times \left(0, T \right],
    \end{align}
    \begin{align}
        u(0, t) = g_{0}(t), \quad u(1, t) &= g_{1}(t) &&\forall t \in [0,T],
    \end{align}
    \begin{align}
        u(x,0) = u_0(x), \quad u_t(x, 0) &= v_0(x) &&\forall x \in  \Omega.
    \end{align}
\end{subequations}
Similarly to the elliptic case we require the coefficient $c \in C^1( \overline{\Omega} \times \left[0,T \right])$ to be bounded by 
\begin{equation}
    \label{eq:wave_coefficient_bounds}
    0 < c_{\min} \leq c(x,t) \leq c_{\max} < \infty \qquad \forall (x,t) \in  \overline{\Omega} \times [0,T].
\end{equation}
We assume the forcing term $ f \in L^2(0,T; L^2(\Omega)) $, the initial displacement $u_0 \in  H^1(\Omega)$, and the initial velocity $v_0 \in L^2(\Omega)$. 
First we assume homogeneous Dirichlet boundary conditions, meaning $g_0 \equiv g_1 \equiv 0$. 
Multiplying the pde (\ref{eq:strong_form_wave}a) by a test function $v \in  C^{\infty}_c (\Omega)$ and integrating by parts gives us the weak formulation: \\ \\
Find $u \in L^2(0,T; H_0^1(\Omega))$, with $ u_t \in L^2(0,T;L^2(\Omega)), u_{tt} \in L^2(0,T;H^{-1}(\Omega)) $, such that
\begin{subequations}
    \label{eq:weak_form_wave}
    \begin{equation}
    \langle u_{tt}(t), v \rangle + a(u(t), v;t) = (f(t),v)_{L^2(\Omega)} \qquad \forall v \in C^{\infty}_c  (\Omega) \quad \text{for a.e. }t\in  (0,T),    
    \end{equation}
    \begin{equation}
        u(\cdot, 0) = u_0, \quad u_t(\cdot, 0) = v_0,
    \end{equation}
\end{subequations}
where 
\begin{equation*}
    a(u(t),v;t) = \int_{\Omega} c(x,t) u_x(x,t) v_x(x) \text{d}x
\end{equation*}
denotes the standard (time-dependent) elliptic bilinear form and $\langle \cdot, \cdot \rangle $ denotes the duality pairing between $H^{-1}(\Omega)$ and $H^1_0(\Omega)$.
Note that both the time derivatives as well as the space derivatives have to be understood in a weak sense. 
It is well known that the problem (\ref{eq:weak_form_wave}) is well posed (see section 7.2 Theorem 3 in \cite{EvansPDE}). Furthermore we even have 
\begin{equation*}
    u \in C^1([0,T]; L^2(\Omega)) \cap C([0,T];H_0^1(\Omega)),
\end{equation*}
which ensures the validity of the initial conditions (\ref{eq:weak_form_wave}b). \\
We can furthermore deduce existence and uniqueness of the weak problem for inhomogeneous Dirichlet boundary conditions, if they are sufficiently regular. Suppose 
for example $g_0 = g(0, \cdot), g_1 = g(1, \cdot)$ for some $g \in C^2([0,T]; H^2(\Omega))$. We can write $u = w + g$, where $w \in C^1([0,T]; L^2(\Omega)) \cap C([0,T];H_0^1(\Omega))$ 
is the unique (weak) solution of 
\begin{align*}
    w_{tt} - (c w_x)_x &= f - g_{tt} + (c g_x)_x   &&\text{in } \Omega \times \left(0,T\right],\\
    w(0,\cdot) = w(1,\cdot) &= 0 && \text{in } [0,T], \\
    w(\cdot, 0) = u_0 - g(\cdot, 0), \quad w_t(\cdot, 0) &= v_0 - g_t(\cdot, 0) && \text{in } \Omega.
\end{align*}
With this we can conclude the problem to be well-posed and continue with the discretization procedure.

% Spacial Discretization ----------------------------------------------------------------------------------------------------------------------------------------------------------------
\section{Variational Formulation and Fully-Discrete-Scheme}
\subsection{Discretization in Space}
We start by discretizing in space. To simplify and clarify calculations we assume the solution to be $u \in C^2([0,T];H^2(\Omega))$. We fix some time $t \in [0,T]$,
the discretization in space is analogous to the elliptical case for each time $t$ (see section \ref{sec:elliptic_var_form} in chapter 1), so we will only briefly recall the steps. \\
First we discretize the domain by choosing a partition $\mathcal{T}_h$ of $\Omega$ using the notation introduced in the last chapter.
We multiply (\ref{eq:strong_form_wave}) by a test function
$v \in V_h$, where $V_h^r(\mathcal{T}_h)$ is the discontinuous finite element space defined (\ref{def:discontinuous_finite_element_space}), multiply over each
element $I \in \mathcal{T}_h$ by parts and finally sum up over all elements. After adding the symmetry and penalty terms necessary for the SIPG variational form we find the semi-discrete SIPG
variational formulation:
\begin{tcolorbox}[mythmstyle]
Find $u_h \in C^2([0,T]; V_h)$ with $u_h(\cdot,0) = \mathcal{I}_h u_0, u_t(\cdot, 0) = \mathcal{I}_h v_0$ such that 
\begin{equation}
    \label{eq:wave_semidiscrete_var_form}
    (\partial_{t}^2 u_{h}(t), v)_{L^2(\Omega)} + b(u_h,v;t) = \ell(v;t) \qquad \forall v \in V_h, t \in [0,T],
\end{equation}
where 
\begin{align*}
    b(u,v;t) & = \sum_{n=0}^N \int_{I_n} c(x,t)u_x(x,t)v_x(x)\, \text{d}x
		-\sum_{n=0}^{N+1} \{c(x_n,t)u_x(x_n, t)\}[v(x_n)] + \{c(x_n,t)v_x(x_n)\}[u(x_n,t)] \\
		& + \sum_{n=0}^{N+1} \texttt{a}_n(t)[u(x_n,t)][v(x_n)],                                   \\
		\ell(v;t)  & = (f(t),v)_{L^2(\Omega)}-g_1(t)c(x_{N+1}^-,t)v_x(x_{N+1}^-) + g_0(t)c(x_0^+,t)v_x(x_0^+) \\
		&+ \texttt{a}_{N+1}(t)g_1(t)v(x_{N+1}^-) + \texttt{a}_0(t) g_0(t) v(x_{0}^+).
\end{align*}
\end{tcolorbox}
Recall $\mathcal{I}_h : C(\overline{\Omega}) \to V_h$ denotes the interpolation operator (\ref{def:interpolation_operator}) (one could also use $L^2$-projection). 
Clearly we can not expect the semi-discrete solution to satisfy continuous initial conditions, therefore the semi-discrete formulation must introduce projected initial condition. \\
Next we can rewrite (\ref{eq:wave_semidiscrete_var_form}) into matrix-vector form. To do so let $\Phi_1,\ldots,\Phi_M$ be a basis of $V_h$ and therefore write 
\begin{equation*}
    u_h(x, t) = \sum_{n=1}^{M} \alpha_n(t) \Phi_n(x).
\end{equation*}
We can plug this into the semi-discrete SIPG variational formulation (\ref{eq:wave_semidiscrete_var_form}), using linearity and the fact that testing against elements in a vector space is equivalent
to testing against basis functions yields the matrix-vector system of ODEs
\begin{equation}
    \label{eq:wave_matrix-vector_semidiscrete_system}
    \textbf{M} \ddot{\mathbf{u}}(t) + \mathbf{B}(t)\mathbf{u}(t) = \mathbf{l}(t) \qquad \forall t \in  [0,T].
\end{equation}
Here $[\mathbf{M}]_{i,j} = (\Phi_j, \Phi_i)_{L^2(\Omega)}$ is the \textit{mass matrix}, which in our case is not time-dependent. \\
$ [\mathbf{B}(t)]_{i,j} = b(\Phi_j, \Phi_i; t)$ is the \textit{stiffness matrix}, similar to the elliptic case but now with a time-dependent coefficient $c$ and
$ [\mathbf{l}(t)]_i = \ell(\Phi_i;t)$ the \textit{load vector}, also time-dependent. Finally $ [\mathbf{u}(t)]_i = \alpha_i(t)$ denotes the solution vector with the time-dependent coefficients
as entries uniquely determining the semi-discrete solution.  \\
\subsection{Discretization in Time}
With this we come to the time-discretization. (\ref{eq:wave_matrix-vector_semidiscrete_system}) is a second order system of ODEs, meaning we can use a chosen time-integration
scheme to discretize the solution in time. One benefit of using discontinuous Galerkin to discretize in space is that the mass matrix 
$\mathbf{M}$ is block-diagonal, since by the construction of the basis 
functions, each element is fully decoupled from the rest, this simplifies solving any linear system with the mass matrix as the system matrix. We can capitalize on this property of DG-FEM by choosing 
an explicit time integration scheme. \\
First we choose a (stable) stepsize $\Delta t > 0$, we write $t_m = m \cdot \Delta t \in [0, T]$ and denote 
\begin{equation*}
    \mathbf{u}_m := \mathbf{u}(t_m), \quad \mathbf{B}_m := \mathbf{B}(t_m), \quad \mathbf{l}_m := \mathbf{l}(t_m).
\end{equation*}
We introduce a second order finite difference quotient to approximate the second order time derivative
\begin{equation*}
    \ddot{\mathbf{u}}_m \approx \frac{ \mathbf{u}_{m+1} - 2 \mathbf{u}_{m} + \mathbf{u}_{m-1}}{\Delta t^2},
\end{equation*}
plugging this into (\ref{eq:wave_matrix-vector_semidiscrete_system}) and reforming yields the fully discrete, explicit leapfrog scheme 
\begin{equation}
    \label{eq:wave_fully_discrete_system}
    \mathbf{M}\mathbf{u}_{m+1} = \Delta t^2 \mathbf{l}_m + (2 \mathbf{M} - \Delta t^2 \mathbf{B}_m) \mathbf{u}_{m} - \mathbf{u}_{m-1}.
\end{equation}
This being a multistep scheme we have to initialize $\mathbf{u}_0, \mathbf{u}_1$. To do so we use the interpolated values of the initial 
displacement to initialize $\mathbf{u}_0$, meaning we can write $ \mathcal{I}_h u_0 = \sum_{n=1}^{M} \alpha_n(0) \Phi_n $ for some coefficients $
\alpha_n(0) \in \mathbb{R}$ and get
\begin{equation*}
    [\mathbf{u}_0]_i = \alpha_i(0).
\end{equation*}
In fact since we have chosen a Lagrangian basis for the space $V_h$, finding the coefficients $\alpha_i(0)$ corresponds to evaluating the to be interpolated function
$u_0$ at the corresponding node $x_i$ on which the basis function $\Phi_i$ is stationed, i.e.\ $\alpha_i(0) = u_0(x_i)$. \\
For $\mathbf{u}_1$ we Taylor-expand 
\begin{equation*}
    \mathbf{u}_1 \approx \mathbf{u}(\Delta t) = \mathbf{u}_0 + \Delta t \mathbf{v}_0 + \frac{\Delta t^2}{2} \ddot{\mathbf{u}}(0) + \mathcal{O}(\Delta t^3),
\end{equation*}
using (\ref{eq:wave_matrix-vector_semidiscrete_system}) we can rewrite $\ddot{\mathbf{u}}(0) = \mathbf{M}^{-1}( \mathbf{l}_0 - \mathbf{B}_0 \mathbf{u}_0)$ and define
\begin{equation*}
    \mathbf{u}_1 :=   \mathbf{u}_0 + \Delta t \mathbf{v}_0 + \frac{\Delta t^2}{2}\mathbf{M}^{-1}( \mathbf{l}_0 - \mathbf{B}_0 \mathbf{u}_0).
\end{equation*}

To guarantee stability of the leapfrog scheme $\Delta t$ has to be chosen small enough to satisfy the CFL condition, here this means specifically that 
the matrix $\mathbf{M} - \frac{\Delta t^2}{4}\mathbf{B}(t)$ has to be symmetric positive definite (SPD) for all times. 
This is equivalent to requiring that 
\begin{equation*}
    \Delta t < 2 \frac{\norm{\mathbf{M}}_2}{\norm{\mathbf{B}(t)}_2}  \qquad \forall t \in [0,T],
\end{equation*}
where $\norm{\mathbf{M}}_2^2 = \sigma_{\max}(\mathbf{M})$ is the spectral matrix norm. Indeed the expression is well-defined, since 
$\mathbf{B}(t)$ itself is SPD for all $t \in [0,T]$. This follows from a slight adaptation of the coercivity proof of the SIPG bilinear form on $V_h$ 
(Theorem \ref{thr:cont_coerc_bilin_form}). For further information on the stability of leapfrog see \cite{groteNumWaveLecture}.