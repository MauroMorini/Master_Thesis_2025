\section{Reproducing Convergence Rates}
To reproduce the expected convergence rates and therein check the validity of our implementation we follow a similar sequence of steps as already presented in section \ref{subsec:conv_rate_ell}.  

\begin{stepscope}
    \begin{step}[Domain]
    We fix the domain $\Omega = (0,10)$, an extended version for better visualization, and set the final time to $T = 10$. 
    \end{step}

    \begin{step}[Exact Solution]
        We decide to approximate the smooth exact solutions 
        \begin{equation*}
            u_1(x,t) = e^{-(x-t+2)^2} ,\qquad u_2(x,t) = \sin(x-t - \pi) \nonumber.
        \end{equation*}
    \end{step}
    
    \begin{step}[Coefficient]
        We fix the smooth coefficients 
        \begin{equation*}
            c_1(x,t) = 1, \quad c_2(x,t) = (\sin(x) + 2)(\cos(t) + 2), \quad c_3(x,t) = \sin(x) + 2, \quad c_4(x,t) = \sin(t) + 2.
        \end{equation*}
        To ensure our exact solution still satisfies the original pde, we calculate the forcing term $f$ in (\ref{eq:elliptic_pde}) symbolically, i.e.\ 
        $f_{i,j} = \partial_t^2 u_i - \partial_x(c_j \partial_x u_i)$.
    \end{step}

    \begin{step}[Boundary Conditions]
        As done in the last chapter we focus on reproducing the convergence rates for Dirichlet boundary conditions on both sides, meaning we directly take the values provided by 
        the exact solution $u$ we start with as boundary conditions, meaning 
        \begin{equation*}
            g(x,t) = u(x,t) \qquad \forall(x,t) \in \{0,10\} \times [0,T].
        \end{equation*}
        In the case of the coefficient $c_1$ we observe that the solution $u \in \{u_1, u_2\}$ satisfies the absorbing boundary condition exactly. When imposing the absorbing
        boundary condition at the upper boundary and a Dirichlet boundary condition ($g(0,t) = u(0,t)$), or a Neumann boundary condition ($g(0,t) = -u_x(0,t)$) at the lower boundary
        we observe the exact same expected convergence rates as in the double Dirichlet case.
    \end{step}

    \begin{step}[Mesh]
        We choose an initial equidistant mesh of meshsize $h = 1$
        \begin{equation*}
           \mathcal{T}_h^{(0)} = \{(0, 1), (1, 2), (2, 3), (3, 4), (4, 5), (5, 6), (6, 7), (7, 8), (8, 9), (9, 10) \}, 
        \end{equation*}
        but the results do not vary for any kind of initial mesh. We fix a number of refinement cycles and obtain a sequence of nested meshes by refining every element each cycle.
        We fix a stepsize scaling factor $\gamma > 0$, such that $\Delta t = \gamma h$ on each of the nested meshes. We have found that $ \gamma  = \frac{1}{50r} $ ensures stability 
        for all chosen coefficients in the case of the equidistant mesh. 
    \end{step}

    \begin{step}[Numerical Approximation]
        Again we only consider $\mathcal{P}^1$- and  $\mathcal{P}^2$-elements, so $r\in \{1,2\}$, 
        and fix a sufficiently large penalization parameter $\sigma = 10(r+1)^2$ to ensure coercivity. We calculate the numerical solution at each timestep using the time-marching scheme 
        (\ref{eq:wave_fully_discrete_system}). Due to the time-dependent bilinear and linear forms we now have to reassemble the stiffness matrix 
        $\mathbf{B}(t)$ and the load vector $\mathbf{l}(t)$ at each timestep anew. This is 
        very costly, especially for growing global degrees of freedom (decreasing meshsize). Luckily for $c_1, c_3, c_4$ we do not in fact have to reassemble everything only for
        $c_2$ we are forced to brute force the iteration.
        In the case of $c \in \{c_1, c_3\}$ the coefficient is not time-dependent at all and therefore neither are the entries of the stiffness matrix, so it holds that 
        \begin{equation*}
            \mathbf{B}(t) = \mathbf{B}(0) \qquad \forall t \in [0, T].
        \end{equation*}
        The load vector 
        On each successive mesh $\mathcal{T}_h^{(l)}$ we find the numerical solution $ u_h^{(l)}$ by solving the system (\ref{eq:fully_discrete_dg_system_elliptic}).
    \end{step}

    \begin{step}[Error Computation]
        We are interested in 
        \begin{enumerate}
            \item the $L^2$-error
                \begin{equation*}
                    \|u - u_h\|_{L^2(\Omega)} = \Big( \int_{\Omega} |u(x) - u_h(x)|^2 \text{d} x \Big)^{1/2} = \Big( \sum_{n=0}^{N}\int_{I_n} |u(x) - u_h(x)|^2 \text{d} x \Big)^{1/2};
                \end{equation*}
            \item the broken $H^1$-error
                \begin{equation*}
                    \|u - u_h\|_{H^1(\mathcal{T}_h )} = \Big( \sum_{n=0}^{N}\int_{I_n} |u^{\prime}(x) - u_h^{\prime}(x)|^2 \text{d} x \Big)^{1/2};
                \end{equation*}
            \item and finally the error in the energy norm 
                \begin{equation*}
                    \|u - u_h\|_{\epsilon} = \Big( \sum_{n=0}^{N} \int_{I_n} c(x)\abs{u^{\prime}(x) - u_h^{\prime}(x)}^2\text{d}x + 
                    \sum_{n=0}^{N+1}{\normalfont \texttt{a}_n}\jump{u(x_n) - u_h(x_n)}^2 \Big)^{1/2}.
                \end{equation*}
        \end{enumerate}
        We compute for each numerical solution $ u_h^{(l)} $ all three named errors and plot them in a plot where both axes are logarithmically scaled.
    \end{step}
\end{stepscope}





\newpage

To test our code we reproduced the expected convergence rates for smooth solutions and coefficients.
We chose the waveguide $\Omega = (0,10)$ as domain and fixed the endtime $T = 10$. Next we defined two exact solutions
\begin{equation}
    u_1(x,t) = e^{-(x-t+2)^2} ,\qquad u_2(x,t) = \sin(x-t - \pi) \nonumber
\end{equation}
and the coefficients
\begin{equation*}
    c_1(x,t) = 1, \quad c_2(x,t) = (\sin(x) + 2)(\cos(t) + 2), \quad c_3(x,t) = \sin(x) + 2, \quad c_4(x,t) = \sin(t) + 2.
\end{equation*}
We imposed Dirichlet boundary conditions on both boundary points based 
on the exact solution and calculated the required forcing term $f_{i,j} = \partial_t^2 u_i - \partial_x(c_j \partial_x u_i)$ such that the chosen 
exact solution $u_i$ solves the problem. In total for $i \in \{1,2\}$ and $j \in \{1,2,3,4\} $ we solved the pde: \\ \\
Find $u \in C^2([0,T];H^2(\Omega))$ such that 
\begin{subequations}
    \begin{equation}
        u_{tt} - (c_j u_x)_x = \partial_t^2 u_i - \partial_x (c_j \partial_x u_i) \qquad \text{in } \Omega \times [0,T],
    \end{equation}
    \begin{equation}
        u(0,\cdot) = u_i(0, \cdot), \quad u(10,\cdot) = u_i(10,\cdot) \qquad \text{in } [0, T],
    \end{equation}
    \begin{equation}
        u(\cdot, 0) = u_i(\cdot, 0), \quad u_t(\cdot, 0) = \partial_t u_i(\cdot, 0) \qquad \text{in } \Omega.
    \end{equation}

\end{subequations}

We considered a sequence of uniform meshes of meshsize $h \in  \{1, \frac{1}{2}, \frac{1}{4}, \frac{1}{8}, \frac{1}{16}, \frac{1}{32}\}$. 
For each mesh we fixed the stepsize $\Delta t = \frac{h}{100(r+1)}$, where $r \in \{1,2\}$ is the polynomial degree of the finite element space.
We calculated the numerical solution $u_h$ for all timesteps using leapfrog in time and SIPG in space as presented in (\ref{eq:wave_fully_discrete_system}).
We calculated the $L^2$-error, the broken $H^1$-error and the error in the energy norm between the exact solution $u_i$ and the numerical solution $u_h$ at 
the final time $T$, meaning we considered the quantities
\begin{equation}
    \| u_i(\cdot, T) - u_h(\cdot, T) \|_{L^2(\Omega)}, \quad \| u_i(\cdot, T) - u_h(\cdot, T) \|_{H^1(\mathcal{T}_h)}, 
    \quad \| u_i(\cdot, T) - u_h(\cdot, T) \|_{h}. \nonumber
\end{equation}
