\section{Numerical Results}

\subsection{Convergence Rates with Exact Solution}
\label{subsec:conv_rate_wave}
To reproduce the expected convergence rates and therein check the validity of our implementation we follow a similar sequence of steps as already presented in section \ref{subsec:conv_rate_ell}.  

\begin{stepscope}
    \begin{step}[Domain]
    We fix the domain $\Omega = (0,10)$, an extended version for better visualization, and set the final time to $T = 10$. 
    \end{step}

    \begin{step}[Exact Solution]
        We decide to approximate the smooth exact solutions 
        \begin{equation*}
            u_1(x,t) = e^{-(x-t+2)^2} ,\qquad u_2(x,t) = \sin(x-t - \pi) \nonumber.
        \end{equation*}
    \end{step}
    
    \begin{step}[Coefficient]
        We fix the smooth coefficients 
        \begin{equation*}
            c_1(x,t) = 1, \quad c_2(x,t) = (\sin(x) + 2)(\cos(t) + 2), \quad c_3(x,t) = \sin(x) + 2, \quad c_4(x,t) = \sin(t) + 2.
        \end{equation*}
        To ensure our exact solution still satisfies the original pde, we calculate the forcing term $f$ in (\ref{eq:elliptic_pde}) symbolically, i.e.\ 
        $f_{i,j} = \partial_t^2 u_i - \partial_x(c_j \partial_x u_i)$.
    \end{step}

    \begin{step}[Boundary and Initial Conditions]
        As done in the last chapter we focus on reproducing the convergence rates for Dirichlet boundary conditions on both sides, we directly take the values provided by 
        the exact solution $u$ we start with as boundary conditions, meaning 
        \begin{equation*}
            g(x,t) = u(x,t) \qquad \forall(x,t) \in \{0,10\} \times [0,T].
        \end{equation*}
        We do the same thing for the initial conditions and in total for $i \in \{1,2\}$ and $j \in \{1,2,3,4\} $ we solve the pde: \\
        Find $u \in C^2([0,T];H^2(\Omega))$ such that 
        \begin{subequations}
            \begin{align*}
                u_{tt} - (c_j u_x)_x &= \partial_t^2 u_i - \partial_x (c_j \partial_x u_i)  &&\text{in } \Omega \times [0,T], \\
                u(0,\cdot) = u_i(0, \cdot), \quad u(10,\cdot) &= u_i(10,\cdot)  &&\text{in } [0, T], \\
                u(\cdot, 0) = u_i(\cdot, 0), \quad u_t(\cdot, 0) &= \partial_t u_i(\cdot, 0)  &&\text{in } \Omega.
            \end{align*}
        \end{subequations}
        In the case of the coefficient $c_1$ we observe that the solution $u \in \{u_1, u_2\}$ satisfies the absorbing boundary condition exactly. When imposing the absorbing
        boundary condition at the upper boundary and a Dirichlet boundary condition ($g(0,t) = u(0,t)$), or a Neumann boundary condition ($g(0,t) = -u_x(0,t)$) at the lower boundary
        we observe the exact same expected convergence rates as in the double Dirichlet case.
    \end{step}

    \begin{step}[Mesh]
        We choose an initial equidistant mesh of meshsize $h = 1$
        \begin{equation*}
           \mathcal{T}_h^{(0)} = \{(0, 1), (1, 2), (2, 3), (3, 4), (4, 5), (5, 6), (6, 7), (7, 8), (8, 9), (9, 10) \}, 
        \end{equation*}
        but the results do not vary for any kind of initial mesh. We fix a number of refinement cycles and obtain a sequence of nested meshes by refining every element each cycle.
        We fix a stepsize scaling factor $\gamma > 0$, such that $\Delta t = \gamma h$ on each of the nested meshes. We have found that $ \gamma  = \frac{1}{50r} $ ensures stability 
        for all chosen coefficients in the case of the equidistant mesh. 
    \end{step}

    \begin{step}[Numerical Approximation]
        Again we only consider $\mathcal{P}^1$- and  $\mathcal{P}^2$-elements, so $r\in \{1,2\}$, 
        and fix a sufficiently large penalization parameter $\sigma = 10(r+1)^2$ to ensure coercivity. We calculate the numerical solution at each timestep using the time-marching scheme 
        (\ref{eq:wave_fully_discrete_system}). Due to the time-dependent bilinear and linear forms we now have to reassemble the stiffness matrix 
        $\mathbf{B}(t)$ and the load vector $\mathbf{l}(t)$ at each timestep anew. This is 
        very costly, especially for growing global degrees of freedom (decreasing meshsize). Luckily for $c_1, c_3, c_4$ we do not in fact have to reassemble everything only for
        $c_2$ we are forced to brute force the iteration.
        In the case of $c \in \{c_1, c_3\}$ the coefficient is not time-dependent at all and therefore neither are the entries of the stiffness matrix, so it holds that 
        \begin{equation*}
            \mathbf{B}(t) = \mathbf{B}(0) \qquad \forall t \in [0, T].
        \end{equation*}
        The majority of the load vector depending on the source term $f$ can also be tested for time-independency. If for example $c = c_1$, 
        the whole system, aside from the boundary conditions, can be reused in every timestep. Since the majority of computation time is required to assemble the system, reusing as much as possible 
        provides a substantial benefit.
    \end{step}

    \begin{step}[Error Computation]
        We consider the error between the exact and numerical solution at the endtime $T$, specifically we consider
        \begin{enumerate}
            \item The $L^2$-error $\displaystyle \|u(T) - u_h(T)\|_{L^2(\Omega)}$;
            \item The broken $H^1$-error $\displaystyle \|u(T) - u_h(T)\|_{H^1(\mathcal{T}_h )}$;
            \item The error in the energy norm $\displaystyle \|u(T) - u_h(T)\|_{\epsilon}$.
        \end{enumerate}
        We compute all these errors for both exact solutions $u_1, u_2$ on each mesh in the sequence of nested (equidistant) meshes.
    \end{step}
\end{stepscope}

\begin{figure}[h!]
    \centering
    
    \begin{minipage}[t]{0.44\textwidth}
        \centering
        \includegraphics[width=\linewidth]{figures/dg_wave_uniform_mesh_exact_sol_errors_P1.jpg}
        \caption{Errors of SIPG in space and leapfrog in time for $P^1$-elements}
        \label{fig:wave_uniform_mesh_error_p1}
    \end{minipage}
    \hfill
    \begin{minipage}[t]{0.44\textwidth}
        \centering
        \includegraphics[width=\linewidth]{figures/dg_wave_uniform_mesh_exact_sol_errors_P2.jpg}
        \caption{Errors of SIPG in space and leapfrog in time for $P^2$-elements}
        \label{fig:wave_uniform_mesh_error_p2}
    \end{minipage}
\end{figure}

The results vary only slightly for different permutations of parameters. In Figures 
\ref{fig:wave_uniform_mesh_error_p1}, \ref{fig:wave_uniform_mesh_error_p2} we show the errors computed for the exact solution $u_2$ and the constant coefficient $c_1$ 
for both $\mathcal{P}^1$- and $\mathcal{P}^2$-elements. We observe a considerable gain in accuracy by employing $\mathcal{P}^2$-elements over $\mathcal{P}^1$-elements. 
It seems that we gain an order of convergence when upgrading to $\mathcal{P}^2$-elements. This is indeed the case in the broken $H^1$-norm and energy norm, but since 
leapfrog limits the error to behave like $\mathcal{O}(\Delta t^2)$ and we coupled $\Delta t = \gamma h$ we expect that for $h \to 0$ even the $L^2$-error curve
will flatten and behave like $\mathcal{O}(h^2)$.  
% TODO: put detailed explanation why O(h^3) is an illusion and reference to theory part

\subsection{Convergence Rates with Absorbing Boundary Conditions}
We now want to test the rate of convergence for a absorbing boundary condition implemented at the upper boundary. We proceed analogously to subsection \ref{subsec:conv_rate_wave} with the 
main difference that now instead of implementing a second (exact) Dirichlet boundary condition at the upper boundary $x = L = 10$ here we introduce an absorbing boundary condition 
(see section \ref{sec:transparent_bc}). As seen previously the absorbing boundary condition is exact in 1d if we assume
\begin{enumerate}
    \item $\text{supp}(f) \in \Omega \times [0,T] $,
    \item $c(x,t) = 1$ for all $x\in \left[L, \infty \right), t \in [0,T]$,
    \item No waves return from infinity, meaning the wave travels exclusively to the right for $x \geq L$.
\end{enumerate}  
Indeed these all hold for both exact solutions $u_1(x,t) = e^{-(x-t+2)^2} , u_2(x,t) = \sin(x-t - \pi)$ as long as we fix $c = c_1 \equiv 1$. Clearly for the other coefficients we used previously
we cannot expect convergence at all due to the systematic error the absorbing boundary condition would introduce.
\begin{figure}[h!]
    \centering
    
    \begin{minipage}[t]{0.44\textwidth}
        \centering
        \includegraphics[width=\linewidth]{figures/dg_wave_uniform_mesh_exact_sol_errors_absorbin_bc_P1.jpg}
        \caption{Errors of SIPG in space and leapfrog in time for $P^1$-elements with absorbing b.c.}
        \label{fig:wave_uniform_mesh_error_absorbing_bc_p1}
    \end{minipage}
    \hfill
    \begin{minipage}[t]{0.44\textwidth}
        \centering
        \includegraphics[width=\linewidth]{figures/dg_wave_uniform_mesh_exact_sol_errors_absorbin_bc_P2.jpg}
        \caption{Errors of SIPG in space and leapfrog in time for $P^2$-elements with absorbing b.c.}
        \label{fig:wave_uniform_mesh_error_absorbing_bc_p2}
    \end{minipage}
\end{figure}
In Figures \ref{fig:wave_uniform_mesh_error_absorbing_bc_p1}, \ref{fig:wave_uniform_mesh_error_absorbing_bc_p2} we observe the expected convergence rates calculated with the exact solution 
$u_2$ and the constant coefficient $c_1 $. \\
Indeed if we consider a coefficient which is not constantly 1 at the boundary $x = L$, then the error does not decrease with decreasing meshsize. 

\begin{figure}[h!]
    \centering
    
    \begin{minipage}[t]{0.44\textwidth}
        \centering
        \includegraphics[width=\linewidth]{figures/dg_wave_uniform_mesh_exact_sol_absorbin_bc_with_non_const_coeff_P2.jpg}
        \caption{Numerical and exact solution with inexact absorbing b.c. and $\mathcal{P}^2$-elements at $T=10$}
        \label{fig:wave_uniform_mesh_absorbing_bc_with_non_const_coeff_p2}
    \end{minipage}
    \hfill
    \begin{minipage}[t]{0.44\textwidth}
        \centering
        \includegraphics[width=\linewidth]{figures/dg_wave_uniform_mesh_exact_sol_errors_absorbin_bc_with_non_const_coeff_P2.jpg}
        \caption{Errors of numerical solution with inexact absorbing b.c. and $\mathcal{P}^2$-elements}
        \label{fig:wave_uniform_mesh_error_absorbing_bc_with_non_const_coeff_p2}
    \end{minipage}
\end{figure}
We choose the coefficient $c_2(x,t) = (\sin(x) + 2)(\cos(t) + 2)$ which is clearly non-constant at $x=L$ and the exact solution $u_2$ and plot the errors. In
Figure \ref{fig:wave_uniform_mesh_error_absorbing_bc_with_non_const_coeff_p2} we observe how the errors are non-decreasing for a decreasing meshsize $h$.
We visualize the exact and the numerical solution at the final time $T=10$ (the time at which the errors are computed)
in Figure \ref{fig:wave_uniform_mesh_absorbing_bc_with_non_const_coeff_p2}. The transparent boundary condition introduces such a heavy
systematic error, that the numerical solution no longer looks anything like the exact solution $u_2$ (plotted in blue).

\subsection{Convergence Rates without Exact Solution}
Starting with an exact solution $u$ and modifying the pde to exactly output $u$ is one way of testing the exactness and convergence rates of the method as done previously. Here we would like 
to explore an alternative method, which doesn't require knowledge of the exact solution itself. \\
Consider a sequence of nested meshes $\{ \mathcal{T}_h^{(m)} \}_{m=0}^M$, where each mesh $\mathcal{T}_h^{(m+1)}$ is obtained by refining every element of the previous mesh $\mathcal{T}_h^{(m)}$, 
by a factor of 2.
The idea is now that we refine the last mesh in the sequence $\mathcal{T}_h^{(M)}$ by a factor of 8, therefore obtaining a very fine mesh $\mathcal{T}_h^{(M+1)}$ on which we compute a numerical solution.
This final numerical solution $u_h^{(M+1)}$ plays the role of the exact solution. We compute the errors 
\begin{equation*}
    \norm{u_h^{(M+1)}(\cdot, T) - u_h^{(m)}(\cdot, T)}_{L^2(\Omega)}, \quad
    \norm{u_h^{(M+1)}(\cdot, T) - u_h^{(m)}(\cdot, T)}_{H^1(\mathcal{T}_h^{(m)})}, \quad 
    \norm{u_h^{(M+1)}(\cdot, T) - u_h^{(m)}(\cdot, T)}_{\epsilon}
\end{equation*}
for all $m \in \{0,\ldots, M\}$.

% TODO: Write evaluation code for dg solution (uh and grad_uh) and extend er


\subsection{Visualization of the SIPG-leapfrog Solution}
In Figures \ref{fig:uniform_mesh_wave_sol_1_p1}-\ref{fig:uniform_mesh_wave_sol_2_p2} 
we visualize the numerical solutions created in \ref{subsec:conv_rate_wave} at the final time $T = 10$ on the once refined mesh
$\mathcal{T}_h^{(1)}$. Clearly the choice of the coefficient $c$ does not influence the picture since we artificially modify the pde to output the desired exact solution. 

\begin{figure}[h!]
    \centering
    
    \begin{minipage}[t]{0.44\textwidth}
        \centering
        \includegraphics[width=\linewidth]{figures/dg_wave_num_sol_1_p1.jpg}
        \caption{$u_1$ with $P^1$-elements at $T = 10$}
        \label{fig:uniform_mesh_wave_sol_1_p1}
        \vspace{10pt}
        \centering
        \includegraphics[width=\linewidth]{figures/dg_wave_num_sol_2_p1.jpg}
        \caption{$u_2$ with $P^1$-elements at $T = 10$}
        \label{fig:uniform_mesh_wave_sol_2_p1}
    \end{minipage}
    \hfill
    \begin{minipage}[t]{0.44\textwidth}
        \centering
        \includegraphics[width=\linewidth]{figures/dg_wave_num_sol_1_p2.jpg}
        \caption{$u_1$ with $P^2$-elements at $T = 10$}
        \label{fig:uniform_mesh_wave_sol_1_p2}
        \vspace{10pt}
        \centering
        \includegraphics[width=\linewidth]{figures/dg_wave_num_sol_2_p2.jpg}
        \caption{$u_2$ with $P^2$-elements at $T = 10$}
        \label{fig:uniform_mesh_wave_sol_2_p2}
    \end{minipage}
\end{figure}

\subsection{Simulating Wave with Piecewise Constant Coefficient }