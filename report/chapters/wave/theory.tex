\section{Error Analysis}
The goal of this section is to derive a priori error estimates to show convergence of the method described in (\ref{eq:wave_fully_discrete_system}).
We will only show convergence for time-independent coefficients, where we will assume a sufficiently regular solution of the problem (\ref{eq:weak_form_wave}). 
Throughout this section we will therefore assume the following: \\

Firstly let $\mathcal{T}_h = \{I_n\}_{n=0}^N$ be a partition of $\Omega$ of meshsize $\displaystyle h = \max_{n=0,\ldots,N} h_n$, let $c \in C^1(\bar{\Omega})$ be a time-independent coefficient, 
let $r\geq 1$ denote a polynomial degree and hence $V_h = V_h^r(\mathcal{T}_h)$ be our finite element space. \\
Fix a sufficiently large constant $\sigma > 0$, such that the (time-independent) bilinear form $\sipgb$ is coercive on $V_h$ (see Theorem \ref{thr:cont_coerc_bilin_form}). 
We will assume that $\sigma$ is big enough to guarantee coercivity of $\sipgb$. Recall that we were
able to extend the bilinear form $\sipgb$ defined on $V_h$ to the space $V = V_h + H^2(\Omega)$ in Section \ref{sec:extension_of_bilin_form}. We find the continuous 
SIPG-variational formulation (with a time-independent coefficient): \\ \\
Find $u \in C^2([0,T];V)$ with $u(0) = u_0, u_t(0) = v_0$ such that 
    \begin{equation}
        \label{eq:wave_sipg_var_form}
        (u_{tt}(t), v)_{L^2(\Omega)} + \sipgb(u(t),v) = \ell(v) \qquad \forall v \in V, t \in [0,T] 
    \end{equation}
This variational form is consistent with the weak formulation (\ref{eq:weak_form_wave}) (with inhomogeneous dirichlet boundary conditions strongly enforced).

\begin{lemma}
    Let $u \in C^2([0,T];H^2(\Omega))$. Then $u$ is a solution of (\ref{eq:weak_form_wave}) if and only if it is a solution of (\ref{eq:wave_sipg_var_form})
\end{lemma}
\begin{proof}
    The proof is similar to Theorem \ref{thr:elliptic_consistency_sipg_variational_formulation}, so we will not go into details.
    \begin{proofstep}
        Suppose $u \in C^2([0,T]; H^2(\Omega))$ satisfies (\ref{eq:weak_form_wave}), then 
        \begin{equation*}
            u_{tt}(t) - (c u_{x}(t))_x = f(t) \qquad \forall t \in [0,T]\text{, a.e.\ in }\Omega.
        \end{equation*}
        We can multiply this by a test function $v \in V$, integrate by parts and generally follow the derivation of the SIPG variational form as presented in detail in 
        section \ref{sec:elliptic_var_form} and find that $u$ satisfies (\ref{eq:wave_sipg_var_form}).
    \end{proofstep}
    \begin{proofstep}
        Similarly suppose $u \in C^2([0,T]; H^2(\Omega))$ satisfies (\ref{eq:wave_sipg_var_form}), then by plugging test functions $v \in C_c^{\infty} \subset V$ into 
        (\ref{eq:wave_sipg_var_form}) we find
        \begin{equation*}
            (u_{tt}(t), v)_{L^2(\Omega)} + a(u(t), v) = (f(t),v)_{\lo},
        \end{equation*}
        which in turn implies
        \begin{equation*}
            u_{tt}(t) - (c u_{x}(t))_x = f(t) \qquad \forall t \in [0,T]\text{, a.e.\ in }\Omega.
        \end{equation*}
        The boundary conditions can be recovered in an analogous manner as done in the proof of Theorem \ref{thr:elliptic_consistency_sipg_variational_formulation}.
    \end{proofstep}
\end{proof}

In the following subsections we will accumulate some necessary tools for the proof of the a priori error estimates.

\subsection{Elliptic Projection}
To show convergence we will require the help of a special kind of projection onto the finite element space, called \textit{elliptic projection}. \\
Since $b_h$ is continuous and coercive on $V_h$, for any $u \in V$ 
\begin{equation*}
    b_h(u,\cdot): V_h \to \R 
\end{equation*}
defines a continuous linear form. $V_h$ is a finite dimensional vector space and the tuple $(V_h, \norm{\cdot}_{\epsilon})$ defines a Hilbert space. With Lax-Milgram we find that for any 
$u \in V$ there exists a unique $w \in V_h$ such that 
\begin{equation*}
    b_h(u,v) = b_h(w,v) \qquad \forall v \in V_h.
\end{equation*}
This leads us to the following definition:

\begin{definition}[Elliptic Projection]
    For each $u \in V$ there exists a unique $\Pi_h u \in V_h$ such that 
    \begin{equation*}
        b_h(u,v) = b_h(\Pi_h u, v) \qquad \forall v \in V_h.
    \end{equation*}
    We call $\Pi_h u$ the \textbf{Elliptic Projection} of $u$ onto $V_h$. Therefore 
    \begin{equation*}
        \Pi_h : V \to V_h
    \end{equation*}
    defines a linear projection operator. 
\end{definition}
The elliptic projection satisfies the following estimates 
\begin{lemma}[Elliptic Projection Estimates]
    Let $r \geq 1$ and $u\in H^{r+1}(\Omega)$. There exist constants $C_1, C_2 > 0$ only dependent on $c,\sigma,\Omega$, but not on $u$ or $h$ such that
    \begin{enumerate}[label=\textnormal{(\roman*)}]
        \item $\displaystyle \norm{u - \Pi_h u}_{\epsilon} \leq C_1 h^r \abs{u}_{H^{r+1}(\Omega)}$,
        \item $\displaystyle \norm{u - \Pi_h u}_{L^2(\Omega)} \leq C_2 h^{r+1} \abs{u}_{H^{r+1}(\Omega)}$.
    \end{enumerate} 
\end{lemma}
\begin{proof}
    First recall the interpolation operator $\mathcal{I}_h^r$ from Definition \ref{def:linear_interp_operator}, which interpolates any target at the element face nodes $x_n$ 
    (plus additional element internal interpolation nodes for $r > 1$). \\
    \begin{proofstep}[Proof of (i)]
        Let $u \in H^{r+1}(\Omega)$, we introduce the interpolant into the error:
        \begin{equation*}
            \norm{u-\Pi_h u}_{\epsilon} = \norm{u- \mathcal{I}_h^r u}_{\epsilon} + \norm{\mathcal{I}_h^r u - \Pi_h u}_{\epsilon}.
        \end{equation*} 
        Because the interpolation nodes of the interpolant lay at the element faces it holds that $\jump{(u- \mathcal{I}_h^r u)(x_n)} = 0$ for all $n \in \{0,\ldots,N+1\}$ and therefore 
        \begin{equation*}
            \enorm{u- \mathcal{I}_h^r u} = \norm{\sqrt{c}(u- \mathcal{I}_h^r u)'}_{L^2(\Omega)} \leq c_{\max} \abs{u- \mathcal{I}_h^r u}_{H^1(\Omega)}
            \leq c_{\max} h^r \abs{u}_{H^{r+1}(\Omega)},
        \end{equation*} 
        where in the last step we have used Lemma \ref{lemma:interp_estimate}. So we have found
        \begin{equation}
            \label{eq:proof_elliptic_proj_estimate_1_first_eq}
            \enorm{u - \Pi_h u} \leq c_{\max} h^r \abs{u}_{H^{r+1}(\Omega)} + \enorm{\mathcal{I}_h^r u - \Pi_h u}.
        \end{equation}
        It remains to estimate the second term. Define $w_h := \mathcal{I}_h^r u - \Pi_h u \in V_h$, since $b_h$ is coercive on $V_h$ 
        with respect to the energy norm we can estimate
        \begin{equation*}
            \enorm{w_h}^2 \leq \frac{1}{C_{\text{coer}}} b_h(w_h, w_h) = \frac{1}{C_{\text{coer}}} 
            \Big( b_h(\mathcal{I}_h^r u - u, w_h) + \underbrace{b_h(u - \Pi_h u, w_h)}_{=0} \Big),
        \end{equation*}
        where the last term is zero by the definition of the elliptic projection.
        Next by using the definition of the bilinear form and Cauchy-Schwarz we can estimate 
        \begin{align*}
            b_h(u - \mathcal{I}_h^r u, w_h) &= 
            \sum_{n=0}^{N} \int_{I_n} c (u - \mathcal{I}_h^r u)' w_h' \text{ d}x 
            - \sum_{n = 0}^{N+1} \avg{c(x_n)(u-\mathcal{I}_h^r u)'(x_n)}\jump{w_h(x_n)} \\
            & - \sum_{n = 0}^{N+1} \avg{c(x_n)w_h'(x_n)} \underbrace{\jump{(u-\mathcal{I}_h^r u)(x_n)}}_{=0}
            + \sum_{n = 0}^{N+1} \texttt{a}_n \jump{w_h(x_n)} \underbrace{\jump{(u-\mathcal{I}_h^r u)(x_n)}}_{=0} \\
            & \leq \sum_{n = 0}^{N} \norm{\sqrt{c}(u-\mathcal{I}_h^r u)'}_{L^2(I_n)} \norm{\sqrt{c}w_h'}_{L^2(\Omega)}
            + \sum_{n = 0}^{N+1} \frac{\avg{c(x_n)(u-\mathcal{I}_h^r u)'(x_n)}}{\sqrt{\texttt{a}_n}} \sqrt{\texttt{a}_n} \jump{w_h(x_n)} \\
            & \leq \enorm{w_h}\Big( \sum_{n=0}^{N} \norm{\sqrt{c}(u-\mathcal{I}_h^r u)'}_{L^2(I_n)}^2
            + \sum_{n=0}^{N+1} \frac{\avg{c(x_n)(u-\mathcal{I}_h^r u)'(x_n)}^2}{\texttt{a}_n}
                \Big)^{1/2} \\
            & \leq \enorm{w_h}\Big( \sum_{n=0}^{N} c_{\max}\abs{u-\mathcal{I}_h^r u}_{H^1(I_n)}^2
            + \sum_{n=0}^{N+1} \frac{\avg{c(x_n)(u-\mathcal{I}_h^r u)'(x_n)}^2}{\texttt{a}_n}
                \Big)^{1/2}.
        \end{align*}
        By again applying Lemma \ref{lemma:interp_estimate} we find 
        \begin{equation*}
            \sum_{n=0}^{N} c_{\max}\abs{u-\mathcal{I}_h^r u}_{H^1(I_n)}^2 \leq c_{\max}h^{2r} \abs{u}_{H^{r+1}(\Omega)}^2.
        \end{equation*}
        Recall that in the proof of Theorem \ref{thr:elliptic_l2_convergence} we have already estimated the terms 
        (see (\ref{eq:elliptic_proof_convergence_in_energy_norm_estimating_interp_error_avg_terms}))
        \begin{equation*}
            \frac{\avg{c(x_n)(u - \mathcal{I}_h^r u)'(x_n)}^2}{\texttt{a}_n} \leq 
            \frac{2 c_{\max}}{\sigma } h^{2r} \Big(|u|_{H^{r+1}(I_{n-1})}^2 + |u|_{H^{r+1}(I_{n})}^2 \Big)
        \end{equation*}
        for interior faces $n \in \{1,\ldots,N\}$ and 
        \begin{equation*}
			\frac{\avg{c(x_{0})(u - \mathcal{I}_h^r u)'(x_{0})}^2}{\texttt{a}_{0} } 
			\leq \frac{4 c_{\max}}{\sigma} h^{2r} |u|^2_{H^{r+1}(I_{0})}, \quad 
			\frac{\avg{c(x_{N+1})(u - \mathcal{I}_h^r u)'(x_{N+1})}^2}{\texttt{a}_{N+1} } 
			\leq \frac{4 c_{\max}}{\sigma } h^{2r} |u|^2_{H^{r+1}(I_{N})}
		\end{equation*}
        for boundary faces. So we have found 
        \begin{equation*}
            \enorm{w_h}^2 \leq \frac{b_h(u - \mathcal{I}_h^r u, w_h) }{C_{\text{coer}}} 
            \leq \enorm{w_h} \frac{\sqrt{c_{\max}(\sigma + 6)}}{\sqrt{\sigma} C_{\text{coer}}} h^{r}\abs{u}_{H^{r+1}(\Omega)}.
        \end{equation*}
    Dividing by $\enorm{w_h}$ on both sides and combining it with (\ref{eq:proof_elliptic_proj_estimate_1_first_eq}) finally yields the desired equation
    \begin{equation*}
        \enorm{u - \Pi_h u} \leq C_1 h^r \abs{u}_{H^{r+1}(\Omega)},
    \end{equation*}
    where $\displaystyle C_1 = c_{\max} + \frac{\sqrt{c_{\max}(\sigma + 6)}}{\sqrt{\sigma} C_{\text{coer}}}$.
    \end{proofstep}
    \begin{proofstep} [Proof of (ii)]
        The proof of (ii) is analogous to the proof of Theorem \ref{thr:elliptic_l2_convergence} so we will not repeat it here.
    \end{proofstep}
\end{proof}

So far we have not needed any restriction on the mesh $\mathcal{T}_h$, this is because we consider only the 1d case. In reality when we consider problems on higher dimensional domains, certain 
regularity assumptions on the mesh are required. These assumptions always appear in the context of a sequence of meshes $ \{\mathcal{T}_h\}_h $. 
We would like that a sequence of meshes we consider for convergence does not contain elements which grow ever smaller in comparison to others which stay big. This yields the following definition.
\begin{definition}
    A sequence of meshes $ \{\mathcal{T}_h\}_h $ is called \textbf{quasi-uniform} if there exists some constant $\kappa > 0$ independent of the meshsize $h$, such that
    \begin{equation*}
        \frac{\max_{n}h_n}{\min_n h_n} \leq \kappa,
    \end{equation*} 
    where $\max_n h_n, \min_n h_n$ denote the maximal, minimal element size per mesh respectively.
\end{definition}
Clearly if we only consider one mesh and not a sequence, every mesh will be quasi-uniform, the key to this definition is that we consider a sequence of meshes. \\

\subsection{Inverse Inequality}
Inequalities like the \textit{Poincaré inequality} or the \textit{Sobolev inequalities } are essential tools in the theory of partial differential equations allowing estimation
of lower order derivatives by higher order derivatives. The opposite is also possible, although only on finite dimensional subspaces.
\begin{lemma}[Inverse Inequality]
    \label{lemma:inverse_inequality}
    Let $I := (a,b) \subset \R$ be an arbitrary (open, nonempty) interval and $r\geq 1$ a polynomial degree, we denote the interval length by $h := b-a$. There exists a constant 
    $C_{\text{inv}}(r) > 0$ only depending on $r$, such that 
    \begin{equation*}
        \norm{v'}_{L^2(I)} \leq C_{\text{inv}} h^{-1} \norm{v}_{L^2(I)}.
    \end{equation*}
\end{lemma} 
\begin{proof}
    \begin{proofstep}
        Recall the reference element $\hat{I} = (-1, 1)$ and define the (affine) element map 
        \begin{equation*}
            F: \hat{I} \to I, \quad \xi \mapsto \frac{a+b}{2} + \frac{h}{2} \xi.
        \end{equation*}
        Next we define the functional 
        \begin{equation*}
            \mathcal{R}: \mathcal{P}^r(\hat{I})\setminus {\{0\}} \to \left[0, \infty \right), \quad \mathcal{R}[\widehat{v}] := \frac{\norm{\widehat{v}^{\,\prime}}_{L^2(\hat{I})}^2}{\norm{\widehat{v}}_{L^2(\hat{I})}^2}.
        \end{equation*}
        Note that $\mathcal{R}$ is homogeneous of degree 0, i.e.\ $\mathcal{R}[\widehat{v}] = \mathcal{R}[\lambda \widehat{v}]$ for any $\lambda > 0$, using this, the finite dimensionality of
        $\mathcal{P}^r(\hat{I})$ and the closedness of the unit ball we have
        \begin{equation*}
            \mathcal{R}_{*}:= \sup_{\widehat{v} \in \mathcal{P}^r(\hat{I})\setminus \{0\}} \mathcal{R}[\widehat{v}] 
            = \max_{\substack{\widehat{v} \in \mathcal{P}^r(\hat{I}) \\ \norm{\widehat{v}}_{L^2(\hat{I})}=1}} \mathcal{R}[\widehat{v}] > 0,
        \end{equation*}
        meaning the supremum is attained, which in turn yields the inequality
        \begin{equation}
            \label{eq:proof_inv_inequality_ineq_on_ref}
            \norm{\widehat{v}^{\, \prime}}_{L^2(\hat{I})}^2 \leq \mathcal{R}_{*} \norm{\widehat{v}}_{L^2(\hat{I})}^2 \qquad  \forall \widehat{v} \in \mathcal{P}^r(\hat{I}).
        \end{equation}
    \end{proofstep}
    \begin{proofstep}
        Let $v \in  \mathcal{P}^r({I})$, we can pull $v$ back to the reference element by defining $\widehat{v}(\xi) := v(F(\xi))$ for all $\xi \in \hat{I}$, or equivalently
        $v(x) = \widehat{v}(F^{-1}(x))$ for all $x\in I$. Clearly $\widehat{v} \in  \mathcal{P}^r(\hat{I})$. \\
        We can now proof the sought inequality by starting on $I$, going to the reference element $\hat{I}$ by substitution, applying (\ref{eq:proof_inv_inequality_ineq_on_ref}) there and 
        coming back to $I$ by substitution once again:
        \begin{align*}
            \norm{v'}_{L^2(I)}^2 &= \int_{I}\abs{v'(x)}^2 \text{d}x = \int_{I} \abs{\widehat{v}^{\,\prime}(F^{-1}(x))}^2 \Big(\frac{2}{h}\Big)^2 \text{d}x
            = \frac{2}{h} \int_{\hat{I}} \abs{\widehat{v}^{\,\prime}(\xi)}^2 \text{d}\xi \\
            &\leq \frac{2\mathcal{R}_{*}}{h} \int_{\hat{I}} \abs{\widehat{v}(\xi)}^2 \text{d}x
            = \frac{4\mathcal{R}_{*}}{h^2} \norm{v}_{L^2(I)}^2 = C_{\text{inv}}^2 h^{-2}\norm{v}_{L^2(I)}^2,
        \end{align*}
        for $C_{\text{inv}}^2 = 4\mathcal{R}_{*}$.
    \end{proofstep}
\end{proof}
By making use of Lemma \ref{lemma:inverse_inequality} we can now estimate the energy norm from above by the $L^2$-norm on the finite element space. In fact one could extend the following lemma 
to show equivalence of the two norms on $V_h$, but we are only interested in the upper bound. 
\begin{lemma}
    Let $r\geq 1$ be a polynomial degree, $\{\mathcal{T}_h\}_h$ be a sequence of quasi-uniform meshes with uniformity constant $\kappa > 0$ and let $V_h = V_h(\mathcal{T}_h)$ be the 
    finite element space. Then there exists a constant $C_{\lambda}(c, \sigma, r, \kappa) > 0$ independent of $h$, such that 
    \begin{equation*}
        \enorm{v}^2 \leq C_{\lambda} h^{-2} \norm{v}_{L^2(\Omega)}^2 \qquad \forall v\in V_h.
    \end{equation*} 
    Furthermore it follows that 
    \begin{equation*}
        b_h(v,v) \leq C_{\text{cont}} C_{\lambda} h^{-2} \norm{v}_{L^2(\Omega)}^2 \qquad \forall v \in V_h.
    \end{equation*}
\end{lemma}
\begin{proof}
    Fix an arbitrary mesh $\mathcal{T}_h$ in the sequence of meshes and denote $h_{\min} = \min_n h_n$ the smallest element size. Due to the quasi-uniformity of the mesh sequence we have
    \begin{equation}
        \label{eq:proof_enorm_equivalence_quasi_uniformity}
        h_{\min}^{-1} \leq \frac{\kappa}{h}.
    \end{equation}
    Let $v \in V_h$, recall the definition of the energy norm 
    \begin{equation*}
        \enorm{v}^2 = \sum_{n=0}^{N_h} \norm{\sqrt{c}v'}^2_{L^2(I_n)} + \sum_{n=0}^{N_h+1} \texttt{a}_n \jump{v(x_n)}^2.
    \end{equation*}
    We can estimate the trace values of the jump terms by applying Lemma \ref{lemma:inv_ineq}:
    \begin{align*}
        \texttt{a}_n \jump{v(x_n)}^2 &= \texttt{a}_n \Big( v(x_n^-) - v(x_n^+)\Big)^2 \leq 2 \texttt{a}_n(\abs{v(x_n^-)}^2 + \abs{v(x_n^+)}^2) \\
        & \leq \frac{2\sigma c_{\max}(r+1)^2}{h_{\min}^2}\Big( \norm{v}_{L^2(I_{n-1})}^2 + \norm{v}_{L^2(I_{n})}^2 \Big)
        \qquad \forall n \in \countset{1}{N_h},
    \end{align*}
    and 
    \begin{equation*}
        \texttt{a}_0\jump{v(x_0)}^2 \leq \frac{\sigma c_{\max}(r+1)^2}{h_{\min}^2} \norm{v}_{L^2(I_0)}^2, \quad 
        \texttt{a}_{N_h+1}\jump{v(x_{N_h+1})}^2 \leq \frac{\sigma c_{\max}(r+1)^2}{h_{\min}^2} \norm{v}_{L^2(I_{N_h})}^2
    \end{equation*}
    for the boundary faces.
    We can apply these estimates together with Lemma \ref{lemma:inverse_inequality} to bound the energy norm
    \begin{equation*}
        \enorm{v}^2 \leq c_{\max} \Big(\sum_{n=0}^{N_h} \norm{v'}_{L^2(I_n)}^2 
        + \frac{3\sigma (r+1)^2}{h_{\min}^2} \norm{v}_{L^2(\Omega)}^2 \Big) 
        \leq c_{\max} h_{\min}^{-2} \Big( C_{\text{inv}}^2 + 3\sigma (r+1)^2\Big) \norm{v}_{L^2(\Omega)}^2
    \end{equation*}
    and with (\ref{eq:proof_enorm_equivalence_quasi_uniformity}) we finally get 
    \begin{equation*}
        \enorm{v}^2 \leq C_{\lambda}h^{-2} \norm{v}_{L^2(\Omega)}^2 \qquad \forall v \in V_h,
    \end{equation*}
    where $C_{\lambda} = c_{\max}\kappa (C_{\text{inv}}^2 + 3(r+1)^2\sigma) $. 
    The estimate on $b_h$ follows directly from the estimate on the energy norm above and the continuity of the bilinear form on $V_h$.
\end{proof}

\subsection{A Priori L2 Error Estimation}
\begin{theorem} [Leapfrog Convergence in $L^2$-norm]
    \label{thr:wave_l2_convergence}
    Let $r\geq 1$ be the polynomial degree, $h_0 > 0$, $\{\mathcal{T}_h\}_{0<h\leq h_0}$ be a quasi-uniform sequence of meshes with uniformity constant $\kappa > 0$ and
    $\Delta t > 0$ (depending on $h$) small enough such that 
    \begin{equation*}
        1 - \frac{C_{\lambda}C_{\text{cont}}\Delta t^2}{4 h^2} > \frac{1}{2}.
    \end{equation*} 
    Let $u \in C^3([0,T];H^{r+1}(\Omega))$ be the (weak) solution of the wave equation (\ref{eq:weak_form_wave}) and 
    $\{u_h^m\}_{m=0}^{M} \subset V_h$ be the sequence of fully discrete solutions from (\ref{eq:wave_fully_discrete_problem}). \\
    Then there exists some constant $\mathcal{C} > 0$ independent of $h,\Delta t, u, T$, such that 
    \begin{equation*}
        \max_{m = 0,\ldots,M} \norm{u(t_m) - u_h^m}_{\lo} \leq \mathcal{C}\sqrt{1 + T^2}(\Delta t^2 + h^{r+1})\norm{u}_{C^3([0,T];H^{r+1}(\Omega))}
    \end{equation*}
\end{theorem}

\begin{proof}
    Due to the length oif this proof and the sheer amount of constants appearing we will not exactly calculate $\mathcal{C}$, to simplify notation we will assume the constant to be
    growing throughout this proof and reuse it as a placeholder.

    \begin{proofstep}
        First define $u^m := u(t_m) \in H^{r+1}(\Omega)$ for all $m \in \countset{0}{M}$.
        By introducing the elliptic projection into the error and using the triangle inequality we can estimate
        \begin{equation}
            \label{eq:proof_wave_l2_convergence_introducing_elliptic_projection_first_estimate}
            \norm{u^m - u_h^m}_{\lo} \leq \norm{u^m - \Pi_h u^m}_{\lo} + \norm{u_h^m - \Pi_h u^m}_{\lo} \qquad \forall m \in \countset{0}{M}.
        \end{equation}
        Define $e^m := u_h^m - \Pi_h u^m \in V_h $. For all $m \in \countset{1}{M-1}$ $u^m$ and  $u_h^m$ satisfy 
        \begin{align}
            \label{eq:proof_wave_l2_convergence_pde_of_cont_solution}
            (\partial_{t}^2 u^m, v)_{\lo} + \sipgb(u^m, v) &= \ell(v) &&\forall v \in V_h, \\
            \label{eq:proof_wave_l2_convergence_pde_of_discrete_solution}
            (\delta^2 u_h^m, v)_{\lo} + \sipgb(u_h^m, v) &= \ell(v) &&\forall v \in V_h
        \end{align}
        respectively.
        We can subtract (\ref{eq:proof_wave_l2_convergence_pde_of_discrete_solution}) - (\ref{eq:proof_wave_l2_convergence_pde_of_discrete_solution}) and find
        \begin{align*}
            0 &= (\partial_t^2 u^m - \delta^2 u_h^m, v)_{\lo} + \sipgb(u^m - u_h^m, v) \\
            & = (\partial_t^2 u^m - \delta^2 \Pi_h u^m + \delta^2 \Pi_h u^m - \delta^2 u_h^m, v)_{\lo} + \sipgb(u^m - \Pi_h u^m + \Pi_h u^m - u_h^m, v) \\
            & = (\partial_t^2 u^m - \delta^2 \Pi_h u^m,v)_{\lo} - (\delta^2 e^m, v)_{\lo} + \underbrace{{\sipgb(u^m - \Pi_h u^m, v)}}_{=0} - \sipgb(e^m, v)
        \end{align*}
        for all $m \in \countset{1}{M-1}$, which leads to the error equation
        \begin{equation}
            \label{eq:proof_wave_l2_convergence_error_equation}
            (\delta^2 e^m, v)_{\lo} + \sipgb(e^m,v) = (r^m, v)_{\lo} \qquad \forall v \in V_h,m\in \countset{1}{M-1}
        \end{equation}
        where $r^m := \partial_t^2 u^m - \delta^2 \Pi_h u^m$, denotes the residual at time $t_m$. 
    \end{proofstep}
\end{proof}
