First we will consider a time-independent elliptic problem. The derivation
of the bilinear form is inspired by 
Chapter 1 in \cite{riviere2008} as well as \cite{georgoulis2011Springer} and \cite{grote2006}
for cross reference.

%---Problem---------------------------------------------------------------
\section{Problem}
We consider the following elliptic model problem:
\begin{equation}
    \label{eq:elliptic_pde}
    -(c(x)u'(x))' = f(x) \qquad \forall x\in \Omega
\end{equation} 
\begin{equation}
    \label{eq:elliptic_pde_bc}
    u(0) = g_0, u(1) = g_1
\end{equation}
Where $\Omega = (0,1)$ is the domain, $g_0, g_1 \in \mathbb{R}$ are
Dirichlet boundary conditions, $f \in L^2(\Omega)$ and $c:\Omega \to \mathbb{R}$
satisfies:
\[
    c_{\min} \leq c(x) \leq c_{\max} \qquad \forall x\in \Omega
\]
for $0 < c_{\min} \leq c_{\max}$.
Multiplying the solution by a test function and integrating by parts over $\Omega$ we get the 
standard weak formulation: \\
Find $u \in H^1(\Omega)$ such that:
\begin{equation}
    a(u,v) = (f,v)_{L^2(\Omega)} \qquad \forall v \in C_c^{\infty}(\Omega)
\end{equation}
Where 
\[
    a:H^1(\Omega) \times H^1(\Omega) \to \mathbb{R}, \qquad (u,v) \mapsto \int_{\Omega} c(x)u'(x)v'(x) \text{d}x
\]  
defines the standard elliptic bilinear form on $H^1(\Omega)$ and 
\[
    (u,v)_{L^2(\Omega)} = \int_{\Omega} uv \,\text{d}x
\]

denotes the $L^2$-inner product.

%---Discretization---------------------------------------------------------
\section{Discretization}
Let $0=x_0 < \cdots < x_{N+1} = 1$, $I_n = (x_n, x_{n+1})$ for $n = 0,\ldots,N$ be the elements and $\mathcal{T}_h = \{I_n\}_{n=0}^N$ a partition
of $\Omega$.
We denote the element length by $h_n = x_{n+1} - x_{n}$ for $n=0,\ldots,N$ and the global meshsize by
$h = \max_{n} h_n$.
Next we define the discontinuous finite element space
\begin{equation}
    V_h^r(\mathcal{T}_h) = \{v \in L^2(\Omega) |\, v|_{I_n} \in \mathcal{P}^r(I_n) \} 
\end{equation}
where $\mathcal{P}^r(I_n)$ denotes the space of polynomials $p:I_n \to \mathbb{R}$ with $\deg(p) = r$
for $r \in \mathbb{N}$. When the context allows it, we will denote the
finite element space with just $V_h$ for simplicity. 
$V_h$ is our final approximation space in which the numerical solution
lays.
We observe that in contrast to a continuous finite element approximation space 
here the resulting solution is a priori discontinuous by construction. 
Furthermore we have here $V_h \not\subset H^1(\Omega)$. 
This is especially apparent in 1d due to the Sobolev embedding $H^1(\Omega) \subset C^0(\Omega)$.
Any discontinuous element of $V_h$ can therefore not be in $H^1(\Omega)$. \\
To proceed we will require the following trace operators:

\begin{definition} 
    Let $v:\Omega \to \mathbb{R}$ be piecewise continuous and let $n \in
    \{1,\ldots,N\}$
    \begin{enumerate}[label=\textnormal{(\roman*)}]
        \item We denote $v(x_n^+) := \lim_{x \searrow x_n} v(x), v(x_n^-) := \lim_{x \nearrow x_n} v(x)$
        the limit from above/below.
        \item We define the \textbf{jump} at $x_n$ as
        \[
        [v(x_n)]:= v(x_n^-) - v(x_n^+)
        \]
        and the \textbf{average} at $x_n$ as
        \[
            \{v(x_n)\}:= \frac{v(x_n^+) + v(x_n^-)}{2}
        \]
        furthermore by convention we set: 
        \[
            [v(x_0)] := -v(x_0^+),\quad [v(x_{N+1})] := v(x_{N+1}^-),\quad 
            \{v(x_0)\}:=v(x_0^+),\quad \{v(x_{N+1})\}:= v(x_{N+1}^-)
        \]  
    \end{enumerate}
\end{definition}

%---Variational Formulation--------------------------------------------------------
\section{Variational Formulation}
To derive the SIPG variational formulation, let $v \in V_h$ be a test
function. For simplicity suppose for now that the coefficient $c \in C^1(\Omega)$ and
the exact solution $u \in H^2(\Omega) \subset C^1(\Omega)$. 
Due to the discontinuity of the test function in contrast to 
continuous FEM we multiply $u$ with $v$ on each element $I_n$
and integrate by parts locally
\begin{equation*}
    \int_{x_n}^{x_{n+1}} fv\, \text{d}x = -\int_{x_n}^{x_{n+1}} (cu')'\, \text{d}x 
    = \int_{x_n}^{x_{n+1}} cu'v'\, \text{d}x 
    - \left. cu'v\right|_{x_n}^{x_{n+1}} \qquad \forall n=0,\ldots,N
\end{equation*}
then sum over all elements
\begin{equation}
    \label{eq:elliptic_sipg_var_form_incomplete}
    (f,v)_{L^2(\Omega)} = \sum_{n=0}^N \int_{I_n} cu'v'\, \text{d}x 
    -\sum_{n=0}^{N+1} [c(x_n)u'(x_n)v(x_n)]
\end{equation}
where we have used that $\sum_{n=0}^N \left. w \right|_{x_n}^{x_{n+1}} = w(x_{N+1}^-) - 
w(x_{N}^+) + w(x_{N}^-) - \cdots - w(x_1^+) + w(x_1^-) - w(x_0^+) = \sum_{n=0}^{N+1} [w(x_n)]$ for any piece-wise continuous function $w$.
\\
By our construction are $c, u'$ continuous on $\Omega$, this means 
\begin{equation}
    \label{eq:id_1_cu_jump_zero}
        [c(x_n)u'(x_n)v(x_n)] = c(x_n)u'(x_n)[v(x_n)] = \{c(x_n)u'(x_n)\}[v(x_n)] \qquad \forall n=0,\ldots,N+1
\end{equation}
and 
\begin{equation}
    \label{eq:id_2_u_jump_zero}
    [u(x_n)] = 0 \qquad \forall n=1,\ldots,N
\end{equation}
To derive the final variational form we will now have to add two additional terms
to (\ref{eq:elliptic_sipg_var_form_incomplete}): \\
\textbf{Step 1:} Firstly we need to symmetrize our currently non-symmetrical right hand side
which will correspond to the SIPG bilinear form.
For that we subtract $\sum_{n=0}^{N+1} \{c(x_n)v'(x_n)\}[v(x_n)]$ on both sides of
(\ref{eq:elliptic_sipg_var_form_incomplete}):
\begin{align*}
    &(f,v)_{L^2(\Omega)}-g_1c(x_{N+1}^-)v(x_{N+1}^-) + g_0c(x_0^+)v(x_0^+) \\
    = &\sum_{n=0}^N \int_{I_n} cu'v'\, \text{d}x 
    -\sum_{n=0}^{N+1} \{c(x_n)u'(x_n)\}[v(x_n)] + \{c(x_n)v'(x_n)\}[u(x_n)]
\end{align*}
note that on the left hand side of the equation we have applied (\ref{eq:id_2_u_jump_zero})
for the interior node contributions of the sum (which therefore vanish), and the boundary condition (\ref{eq:elliptic_pde_bc}) 
ensuring the left hand side to be soley dependent on $v$.\\
\textbf{Step 2:} The bilinear form we seek to create will be defined on $V_h\times V_h$
meaning it will intake discontinuous functions. In particular the numerical
solution will be a discontinuous function wheras the exact solution is continuous.
To counterweigh this discrepancy we need to integrate a penalization mechanism, seeking to 
minimize discontinuous behaviors. Technically speaking this penalization term 
will guarantee coercivity of the bilinear form. \\
Let $\sigma > 0$ constant, we define:
\begin{equation*}
    \texttt{c}(x_n) := 
    \begin{cases}
        \max(c(x_n^+), c(x_n^-)), &n=1,\ldots,N \\
        c(x_n^+), &n=0 \\
        c(x_n^-), &n=N+1
    \end{cases},
    \qquad \texttt{h}(x_n) :=
    \begin{cases}
        \min(h_n, h_{n-1}), &n=1,\ldots,N \\
        h_n, &n\in \{0, N+1\}
    \end{cases}
\end{equation*}
with this we define our penalization parameter 
\begin{equation}
    \label{def:penalization_function}
    \texttt{a}(x_n) := \frac{\sigma \texttt{c}(x_n)}{\texttt{h}(x_n)} > 0 \qquad \forall n=0\ldots,N+1  
\end{equation}
Similarly to Step 1 we can now add the term $\sum_{n=0}^{N+1} \texttt{a}(x_n)[u(x_n)][v(x_n)]$
on both sides of (\ref{eq:elliptic_sipg_var_form_incomplete}) and get the final
\textit{discrete} SIPG variational formulation.\\
Find $u_h \in V_h$ such that:
\begin{equation}
    \label{eq:discrete_var_form_elliptic}
    b(u_h, v) = \ell(v), \qquad \forall v\in V_h
\end{equation}
where
\begin{align*}
    b(u,v) &= \sum_{n=0}^N \int_{I_n} cu'v'\, \text{d}x 
    -\sum_{n=0}^{N+1} \{c(x_n)u'(x_n)\}[v(x_n)] + \{c(x_n)v'(x_n)\}[u(x_n)]
    +\sum_{n=0}^{N+1} \texttt{a}(x_n)[u(x_n)][v(x_n)] \\
    \ell(v) &= (f,v)_{L^2(\Omega)}-g_1c(x_{N+1}^-)v(x_{N+1}^-) + g_0c(x_0^+)v(x_0^+)
    + \texttt{a}(x_{N+1})g_1v(x_{N+1}) + \texttt{a}(x_{0})g_0v(x_{0})
\end{align*}

%---Boundary Conditions----------------------------------------------------------
\section{Boundary Conditions}
By subtracting the term $\sum_{n=0}^{N+1} \{c(x_n)v'(x_n)\}[v(x_n)]$ on both sides
of (\ref{eq:elliptic_sipg_var_form_incomplete}) we \textit{weakly} imposed the Dirichlet
boundary conditions into the variational form. This stands in contrast to how boundary 
conditions are usually imposed in continuous FEM. Indeed one could also impose them strongly,
meaning we could define 
\begin{equation*}
    V_h^r(\mathcal{T}_h) = \{v \in L^2(\Omega) |\, v|_{I_n} \in \mathcal{P}^r(I_n), v(x_0)=g_0, v(x_{N+1})=g_1 \}
\end{equation*}
but this soley as a side note, we will continue to work with purely weakly imposed 
boundary conditions.  

%---Matrix-Vector System---------------------------------------------------------
\section{Matrix-Vector System}
We will now in derive the fully discrete Matrix-Vector system given by
the variational form (\ref{eq:discrete_var_form_elliptic}). To do so let
$r \in \mathbb{N}$ denote the polynomial degree and consequently the element degree of freedom. Note that in this thesis we will only consider global polynomial degrees.
Next let $\{\Phi_0,\ldots,\Phi_M\}$ be a basis of $V_h$, where $M = \dim(V_h)$.
We can represent the sought Galerkin approximation $u_h = \sum_{j=0}^{M} \alpha_j \Phi_j\in V_h$ for coefficients 
$\alpha_j \in \mathbb{R}$. Then (\ref{eq:discrete_var_form_elliptic}) is equivalent to:
\begin{equation*}
    \sum_{j=0}^{M} \alpha_j b(\Phi_j, \Phi_i) = \ell(\Phi_i) \qquad \forall i=0,\ldots,M
\end{equation*}
which corresponds to the system:
\begin{equation}
    \label{eq:fully_discrete_dg_system_elliptic}
    \textbf{Bu} = \textbf{l}
\end{equation}
for $ \textbf{B} \in \mathbb{R}^{M\times M}, [\textbf{B}]_{i,j} = b(\Phi_j, \Phi_i), 
\textbf{u} \in \mathbb{R}^M, [\textbf{U}]_j = \alpha_j, 
\textbf{l}\in\mathbb{R}^M, [\textbf{l}]_j = \ell(\Phi_j)$.

%---Basis-of-FE-Space------------------------------------------------------------
\section{Basis of Finite Element Space}
%---Existence-------------------------------------------------------------------------
\section{Existence of Discrete Solution}
Firstly we will recall some basic definitions:
\begin{definition} Let $V$ be a normed vector space and $b:V\times V \to \mathbb{R}$
    be a bilinear form.
    \begin{enumerate}[label=\textnormal{(\roman*)}]
        \item We say $b$ is \textbf{continuous} if $\exists C_{\text{cont}}>0$, such that
        \[
            |b(u,v)|\leq C_{\text{cont}}\, \|u\|\, \|v\| \qquad \forall u,v \in V
        \]
        \item We say $b$ is \textbf{symmetric} if 
        \[  
            b(u,v) = b(v,u) \qquad \forall u,v \in V
        \]
        \item We say $b$ is \textbf{coercive} if $\exists C_{\text{coer}}>0$, such that
        \[
            b(u,u)\geq C_{\text{coer}}\, \|u\|^2 \qquad \forall u \in V
        \]
    \end{enumerate}
    
\end{definition}


Since (\ref{eq:discrete_var_form_elliptic}) corresponds to the finite dimensional
system (\ref{eq:fully_discrete_dg_system_elliptic}) uniqueness and existence of a solution
are equivalent. 
The bilinear form $b$ is \textit{symmetric} by construction 
the goal of this section is to show that $b$ is also \textit{coercive}. 
From the coercivity of $b$ it will follow that the matrix $B$ in (\ref{eq:fully_discrete_dg_system_elliptic})
is positive definite and hence invertible, which means there exists a (unique) solution
of (\ref{eq:discrete_var_form_elliptic}).
\begin{lemma}
    Let $V = \text{span}(\varphi_1,\ldots,\varphi_N)$ be a finite dimensional 
    normed vector space with $\dim(V) = N\in \mathbb{N}$ and let
    $b:V \times V \to \mathbb{R}$ be a symmetric, coercive bilinear form,
    then the matrix ${[\textbf{B}]}_{i,j} = {[b(\varphi_j, \varphi_i)]}_{i,j}\in \mathbb{R}^{N\times N}$
    is symmetric positive definite.
\end{lemma}
\begin{proof}
    Clearly $\textbf{B}$ is symmetric. \\
    Let $\textbf{v}=(v_1,\ldots,v_N)\in \mathbb{R}^N$ then $v = \sum_{i=1}^{N}
    v_i \varphi_i\in V$ and we have:
    \[
        \textbf{v}^{T}\textbf{B}\textbf{v} = \sum_{i,j=1}^{N}v_i v_j b(\varphi_j,\varphi_i) = b(v,v) \geq
        C_{\text{coer}} \|w\|^2
    \]
    where we have used the biliniearity and the coercivity of $b$.
\end{proof}
Next we will require a usefull tool often used in FEM proofs to bound
a boundary integral with the integral over the interior domain. These kind
of inequalities are in the literature often called \textit{inverse (trace) inequalities}
and are in essence trace inequalities on finite dimensional subspaces.
We will here rely on a result and proof as presented in \cite{warburtonHesthaven2003ineq}.

\begin{lemma}[Inverse inequality]
    \label{lemma:inv_ineq}
    Let $r\in \mathbb{N}$ be the polynomial degree, 
    $a,b\in \mathbb{R}$ with $a<b$ and let $\mathcal{P}^r([a,b])$ denote the 
    space of polynomials of degree $r$ defined on $[a,b]$. \\
    For any $v\in \mathcal{P}^r((a,b))$ we have:
    \begin{enumerate}
        \item $|v(a)|^2 \leq \frac{{(r+1)}^2}{|b-a|}\|v\|_{L^2([a,b])}^2$ 
        \item $|v(b)|^2 \leq \frac{{(r+1)}^2}{|b-a|}\|v\|_{L^2([a,b])}^2$
    \end{enumerate}
\end{lemma}
\begin{proof}
    We will prove the statements first for the reference element $\hat{I} = [-1,1]$ and then
    use a scaling argument to show the general case by applying a simple substitution. \\
    \begin{proofstep}[Setup]
        We will make 
        use of the Legendre orthonormal basis of $\mathcal{P}^r(\hat{I})$:
        Let $P_0,\ldots,P_r$ denote the Legegndre polynomials on  $\mathcal{P}^r(\hat{I})$. 
        Recall the following well known facts (see for example \cite{quarteroniNumericalMathematicSpringer2007}): 
        \begin{enumerate}
            \item $\{P_0,\ldots,P_r\}$ form an orthogonal basis of $\mathcal{P}^r(\hat{I})$ under the
            $L^2(\hat{I})$ inner product. Meaning:
            \[
            \text{span}(P_0,\ldots,P_r) = \mathcal{P}^r(\hat{I}),\quad \int_{-1}^{1} P_i P_j \,\text{d}\xi = 
            \begin{cases}
                \frac{2}{2i + 1}, &\text{for } i=j \\
                0, &\text{for } i\neq j   
            \end{cases} 
            \]    
            \item $P_i(1) = 1, P_i(-1) = {(-1)}^i, \qquad \forall i = 0,\ldots,r$
        \end{enumerate}
        Let $\psi_i = \frac{(2i + 1)P_i}{2}$ for $i=0,\ldots,r$ denote the normed 
        basis function. Clearly we now have 
        \[
            \psi_i(-1) = {(-1)}^i\sqrt{\frac{2i+1}{2}}, \qquad \psi_i(1) = \sqrt{\frac{2i+1}{2}}, 
            \qquad \int_{-1}^{1}\psi_i \psi_j \text{d}\xi = \delta_{i,j}, \qquad \forall i = 0,\ldots,r
        \]
        where $\delta_{i,j} = 
        \begin{cases}
            1, &\text{for } i = j\\
            0, &\text{for } i\neq j   
        \end{cases}$, and hence $\{\psi_0,\ldots,\psi_r\}$ form an orthonomal basis.
    \end{proofstep} 
    \begin{proofstep}[Proof on reference element]
        For any $v\in \mathcal{P}^r(\hat{I})$
        there exist coefficients $v_0,\ldots,v_r \in \mathbb{R}$, such that 
        $v = \sum_{i=0}^{r}v_i \psi_i$. By applying Cauchy-Schwarz we find
        \[
            |v(-1)|^2 = \Big|\sum_{i=0}^{r}v_i \psi_i(-1)\Big|^2 \leq \Big(\sum_{i=0}^{r} v_i^2 \Big)\Big(\sum_{i=0}^{r} \psi_i(-1)^2 \Big)
            = \Big(\sum_{i=0}^{r} v_i^2 \Big)\Big(\sum_{i=0}^{r} \frac{2i+1}{2} \Big)
            = \Big(\sum_{i=0}^{r} v_i^2 \Big)\frac{(r+1)^2}{2}
        \]
        and finally the orthonormality of the $\psi_i$ yields
        \[
            \frac{(r+1)^2}{2}\sum_{i=0}^{r} v_i^2  = \frac{(r+1)^2}{2}\sum_{i,j=0}^{r} v_i v_j \delta_{i,j}
            = \frac{(r+1)^2}{2} \|v\|_{L^2(\hat{I})}^2
        \]
        This yields the first inequality for the reference element. The second inequality
        can be proven analogously.
    \end{proofstep} 
    \begin{proofstep}[Scaling argument]
        Now we assume that $v \in \mathcal{P}^r([a,b])$.
        Using the affine (element) map 
        \[
            F:[-1,1] \to [a,b], \xi \mapsto \frac{a + b}{2} + \frac{|b-a|}{2}\xi
        \] 
        we can pull $v$ back to the reference element by defining
        $\widehat{v}(\xi):=v(F(\xi))$ for all $\xi \in \hat{I}$.
        Clearly $\widehat{v} \in \mathcal{P}^r(\hat{I})$ hence, by Step 2 we obtain
        \[
            |v(a)|^2 = |\widehat{v}(F^{-1}(a))|^2 = |\widehat{v}(-1)|^2 
            \leq \frac{(r+1)^2}{2}\int_{-1}^{1}\widehat{v}(\xi)^2 \text{d}\xi =
            \frac{(r+1)^2}{2}\frac{2}{|b-a|}\|v\|_{L^2([a,b])}^2 
        \] 
        where in the last equality we have applied a change of variable $x=F(\xi)$ to the integral.
        Applying the same line of reasoning to $|v(b)|^2$ yields both inequalities
        and so we are done.
    \end{proofstep}
    \\
\end{proof}
\medskip
Recall the in previous sections established notations and let $r\in \mathbb{N}$ denote the polynomial 
degree and $V_h^r(\mathcal{T}_h)$ be the discrete subspace.
\begin{definition}
    We define the \textbf{energy norm} on $V_h$ by 
    \begin{equation}
        \label{def:energy_norm}
        \|v\|_h^2 := \sum_{n=0}^{N} \int_{I_n} c(x)v'(x)^2\text{d}x + \sum_{n=0}^{N+1}{\normalfont\texttt{a}}(x_n)[v(x_n)]^2 
    \end{equation}
    where {\normalfont\texttt{a}} denotes the penalization function in (\ref{def:penalization_function}).
\end{definition}
\begin{lemma}
    $\|\cdot\|_h$ defines a norm on $V_h$.
\end{lemma}
\begin{proof}
    Clearly we have $\|\lambda v\|_h =|\lambda|\, \|v\|_h$ for all $\lambda \in \mathbb{R}, v\in V_h$. \\ \\
    By definition we have $\texttt{a}, c>0$ and by extension $\|v\|_h \geq 0$ for all
    $v\in V_h$. Suppose now that $\|v\|_h = 0$ for some $v\in V_h$, then we must have $v|_{I_n} \equiv \text{const}$ 
    and $[v(x_n)] = 0$ for all $n$. So $v$ must be constant on all elements and have a jump of zero at the element boundaries.
    These two facts combined imply that $v$ is constant on all of $\Omega$. By the convention that the jump 
    of any function is zero at the boundary nodes of $\Omega$ it immediately follows that $v=0$. Clearly
    $\|0\|_h=0$, therefore $\|\cdot\|_h$ is positive definite. \\ \\
    Using $[v(x_n) + w(x_n)] = [v(x_n)] + [w(x_n)] \quad \forall v,w \in V_h, n = 0,\ldots,N+1$ we find
    \begin{align*}
        \|v+w\|_h &\leq \Big(\sum_{n=0}^{N} (\|\sqrt{c}v'\|_{L^2(I_n)}+\|\sqrt{c}w'\|_{L^2(I_n)})^2 + 
        \sum_{n=0}^{N+1} (\sqrt{\texttt{a}(x_n)}([v(x_n)] + [w(x_n)]))^2\Big)^{1/2} \\
        &\leq \|v\|_h + \|w\|_h
    \end{align*}
    where in the last inequality we have used the triangle inequality of the euclidian vector norm on $\mathbb{R}^{2N+3}$, with the vector given
    as 
    
    \[
    \textbf{v} = \big[\|\sqrt{c}v'\|_{L^2(I_0)},\ldots,\|\sqrt{c}v'\|_{L^2(I_N)}, \sqrt{\texttt{a}(x_0)}[v(x_0)],\ldots,\sqrt{\texttt{a}(x_{N+1})}[v(x_{N+1})]\big]^T
    \]
    this shows the triangle inequality for $\|\cdot\|_h$ and hence it is a norm.
\end{proof}

\begin{theorem}
    For any polynomial degree $r\in \mathbb{N}$ the bilinear form $b$ in (\ref{eq:discrete_var_form_elliptic})
    is coercive and continuous on $V_h^r(\mathcal{T}_h)$.
\end{theorem}
\begin{proof}
    \begin{proofstep}[coercivity]
        Let $w\in V_h$. Note that 
        \begin{equation}
            b(w,w) = \|w\|_h^2 - 2\sum_{n=0}^{N+1}\{c(x_n)w'(x_n)\}[w(x_n)]
        \end{equation}
        To derive the coercivity of $b$ we will estimate the term $2\sum_{n=0}^{N+1}\{c(x_n)w'(x_n)\}[w(x_n)]$
        from above using Lemma \ref{lemma:inv_ineq} and additional smaller tools:\\
        Using the general fact $2y_1y_2 \leq y_1^2 + y_2^2$
    \end{proofstep}
\end{proof}



